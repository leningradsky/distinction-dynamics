%==============================================================================
% SPACETIME DIMENSION FROM TRIADIC NECESSITY
% Why 3+1 Dimensions
% Part III, Chapter 1
%==============================================================================

\chapter{Spacetime Dimension from Triadic Necessity}\label{ch:dimension}

\epigraph{Space has three dimensions because distinction requires three.}{---}

\vnew{This chapter derives D=3+1 from DD principles.}

\section{The Problem}

Why does spacetime have 3 spatial dimensions and 1 time dimension?

Standard physics: this is a brute fact, input to the theory.

DD claim: this follows from triadic necessity.

\section{Review: Triadic Necessity}

From Part I, we established:
\begin{itemize}
    \item Dyad is ontologically insufficient (Chapter~\ref{ch:dyad})
    \item Triad is the minimal self-sufficient structure
    \item $\mathrm{rank}(\Delta) \geq 2$ for dynamics
\end{itemize}

\section{Spatial Dimensions from Spectral Realization}

\begin{theorem}[Three Spatial Dimensions]
The triad $(A, B, C)$ requires three independent spectral directions for its realization.
\end{theorem}

\begin{proof}[Argument]
\begin{enumerate}
    \item The triad has three elements with pairwise distinctions: $\delta_{AB}, \delta_{BC}, \delta_{CA}$.
    \item Each distinction requires a direction in which it can be measured.
    \item For three independent distinctions, we need three linearly independent directions.
    \item These directions constitute a 3-dimensional space.
\end{enumerate}
\end{proof}

\subsection{Group-Theoretic Argument}

\begin{theorem}[SU(3) Requires 3D]
The group $SU(3)$ — the minimal non-abelian group of rank $\geq 2$ — has:
\begin{itemize}
    \item 3 colors (fundamental representation)
    \item 8 generators (adjoint representation)
    \item 2 Casimir operators (rank 2)
\end{itemize}

The spectral realization requires 3 independent eigenvalues $(\lambda_1, \lambda_2, \lambda_3)$ with $\sum \lambda_i = 0$.

This naturally embeds in 3D space.
\end{theorem}

\section{Time from Phase Evolution}

\begin{theorem}[Time from Non-Eliminable Phases]
In the triad, phase differences between states cannot be globally eliminated.
\end{theorem}

\begin{proof}[Argument]
\begin{enumerate}
    \item SU(2) (dyad) has rank 1: one phase, globally eliminable.
    \item SU(3) (triad) has rank 2: two independent phases.
    \item Phase evolution: $\phi_{ij}(t) = (E_i - E_j)t$.
    \item With three eigenvalues, two independent phase evolutions exist.
    \item These cannot all be eliminated by global transformation.
    \item Non-eliminable phase $\Rightarrow$ intrinsic time.
\end{enumerate}
\end{proof}

\section{Why 3+1 and Not Other Signatures}

\subsection{Why Not 2+1?}

\begin{itemize}
    \item 2 spatial dimensions = dyadic (rank 1)
    \item No stable atoms, no chemistry
    \item No gravity propagation (GR trivial in 2+1)
\end{itemize}

\subsection{Why Not 4+1 or Higher?}

\begin{itemize}
    \item More than 3 spatial dimensions: unstable orbits (Ehrenfest)
    \item No stable planetary systems
    \item No stable atoms (Coulomb potential wrong scaling)
\end{itemize}

\subsection{Why Not 3+2 (Two Time Dimensions)?}

\begin{itemize}
    \item Closed timelike curves everywhere
    \item No well-posed initial value problem
    \item Causality violation
\end{itemize}

\section{The DD Argument}

\begin{theorem}[Uniqueness of 3+1]
3+1 is the unique signature satisfying:
\begin{enumerate}
    \item Triadic spatial necessity (from $\mathrm{rank}(\Delta) = 2$)
    \item Causal structure (from phase non-eliminability)
    \item Physical stability (atoms, orbits, propagation)
\end{enumerate}
\end{theorem}

\section{Connection to Fisher-Ricci}

The Fisher metric on the triadic state space is:
\begin{equation}
g_{ij}(\theta) = \mathbb{E}[\partial_i \log p \cdot \partial_j \log p]
\end{equation}

For three states with two independent parameters ($\mathrm{rank} = 2$):
\begin{equation}
\dim(\Theta) = 3 - 1 = 2 \quad\text{(parameter space)}
\end{equation}

But the \textbf{embedding} space (where states live) is 3-dimensional.

Time emerges from the \textbf{flow} on this manifold:
\begin{equation}
\partial_t g_{ij} = -2 \mathrm{Ric}_{ij} + \ldots
\end{equation}

\section{Anthropic vs DD}

\begin{center}
\begin{tabular}{@{}ll@{}}
\toprule
\textbf{Anthropic} & \textbf{DD} \\
\midrule
3+1 because observers exist & 3+1 because triadic structure \\
Selection effect & Necessity \\
Many universes, we're in right one & Only one possibility \\
Post hoc explanation & A priori derivation \\
\bottomrule
\end{tabular}
\end{center}

DD does not invoke observers. 3+1 follows from the mathematics of distinction.

\section{Summary}

\begin{center}
\fbox{\parbox{0.9\textwidth}{
\textbf{Dimension Theorem}

Space has 3 dimensions because:
\begin{itemize}
    \item Triadic structure requires 3 independent spectral directions
    \item $SU(3)$ is the minimal rank-2 non-abelian group
\end{itemize}

Time has 1 dimension because:
\begin{itemize}
    \item Phase evolution is intrinsically 1-dimensional
    \item More time dimensions violate causality
\end{itemize}

3+1 is \textbf{necessary}, not contingent.
}}
\end{center}
