%==============================================================================
% QUANTUM MECHANICS FROM FISHER INFORMATION
% Schrödinger Equation as Information Extremum
% Part III, Chapter 4
%==============================================================================

\chapter{Quantum Mechanics from Fisher Information}\label{ch:qm-fisher}

\epigraph{The Schrödinger equation is the Euler-Lagrange equation of Fisher information.}{---B. Roy Frieden}

\vnew{This chapter is new in DD v2.0. Derives QM from information geometry.}

\section{The Frieden Program}

B. Roy Frieden (1998, 2004) showed that fundamental physical laws can be derived from a single variational principle:

\begin{center}
\fbox{Minimize Fisher Information}
\end{center}

This is not speculation — it is a published, peer-reviewed result.

\section{Fisher Information Functional}

\begin{definition}[Fisher Information]
For a probability amplitude $\psi(x)$ with $p(x) = |\psi(x)|^2$:
\begin{equation}
I[\psi] = 4 \int |\nabla\psi|^2 \, dx
\end{equation}
\end{definition}

\begin{remark}
This is the kinetic term in quantum mechanics disguised as information.
\end{remark}

\section{The Variational Principle}

\begin{definition}[Bound Information]
The physical constraint (``bound information'') is:
\begin{equation}
J[\psi] = \int V(x) |\psi|^2 \, dx
\end{equation}
where $V(x)$ is the potential.
\end{definition}

\begin{principle}[Extreme Physical Information (EPI)]
Physical laws emerge from extremizing:
\begin{equation}
K = I - J = 4\int |\nabla\psi|^2 dx - \int V|\psi|^2 dx
\end{equation}
subject to normalization $\int |\psi|^2 dx = 1$.
\end{principle}

\section{Derivation of Schrödinger Equation}

\begin{theorem}[Frieden]\label{thm:schrodinger}
The Euler-Lagrange equation for the EPI functional $K$ is the time-independent Schrödinger equation.
\end{theorem}

\begin{proof}
The Lagrangian density is:
\[
\mathcal{L} = 4|\nabla\psi|^2 - V|\psi|^2 - E|\psi|^2
\]
where $E$ is a Lagrange multiplier for normalization.

Euler-Lagrange equation:
\[
\frac{\partial \mathcal{L}}{\partial \psi^*} - \nabla \cdot \frac{\partial \mathcal{L}}{\partial(\nabla\psi^*)} = 0
\]

Computing:
\begin{align}
\frac{\partial \mathcal{L}}{\partial \psi^*} &= -V\psi - E\psi \\
\frac{\partial \mathcal{L}}{\partial(\nabla\psi^*)} &= 4\nabla\psi
\end{align}

Therefore:
\[
-V\psi - E\psi - 4\nabla^2\psi = 0
\]

Rewriting with $\hbar^2/2m = 1$ (natural units):
\[
-\frac{\hbar^2}{2m}\nabla^2\psi + V\psi = E\psi
\]

This is the \textbf{time-independent Schrödinger equation}.
\end{proof}

\section{Time-Dependent Case}

For time evolution, we use:
\begin{equation}
S = \int_0^T \left(I[\psi(t)] - J[\psi(t)]\right) dt
\end{equation}

The Euler-Lagrange equation gives:
\begin{equation}
i\hbar \frac{\partial\psi}{\partial t} = -\frac{\hbar^2}{2m}\nabla^2\psi + V\psi
\end{equation}

The \textbf{full Schrödinger equation}.

\section{DD Interpretation}

\begin{center}
\fbox{\parbox{0.9\textwidth}{
\textbf{Quantum mechanics is the dynamics of minimal distinguishability.}

The wavefunction $\psi$ encodes the probability of distinguishing states. The Schrödinger equation is the flow that minimizes the ``cost'' of distinction while respecting physical constraints.
}}
\end{center}

In DD terms:
\begin{align}
|\psi|^2 &= p(x|\theta) && \text{(probability of distinction)} \\
I[\psi] &= g_{ij} \dot{\theta}^i \dot{\theta}^j && \text{(Fisher metric)} \\
V(x) &= \text{constraint on } \Delta && \text{(bound information)}
\end{align}

\section{Uncertainty Principle}

\begin{theorem}[Cramér-Rao Bound]
For any unbiased estimator $\hat{\theta}$ of parameter $\theta$:
\begin{equation}
\mathrm{Var}(\hat{\theta}) \geq \frac{1}{I(\theta)}
\end{equation}
\end{theorem}

\begin{corollary}[Heisenberg from Fisher]
The Heisenberg uncertainty principle is a special case of the Cramér-Rao bound applied to position and momentum as conjugate parameters.
\end{corollary}

\textbf{DD translation:} Uncertainty = minimum cost of distinction.

\section{Comparison with Standard QM}

\begin{table}[h]
\centering
\begin{tabular}{@{}ll@{}}
\toprule
\textbf{Standard QM} & \textbf{DD/Fisher QM} \\
\midrule
$\psi$ is primitive & $\psi$ encodes distinctions \\
Schrödinger is postulated & Schrödinger is derived \\
Uncertainty is fundamental & Uncertainty = Cramér-Rao \\
$\hbar$ is a constant & $\hbar$ sets distinction scale \\
\bottomrule
\end{tabular}
\end{table}

\section{Key Result}

\begin{theorem}[QM from DD]
Quantum mechanics is the extremal dynamics of distinction in the presence of constraints.

The Schrödinger equation is:
\begin{equation}
\delta I - \delta J = 0 \quad\Rightarrow\quad i\hbar\partial_t\psi = H\psi
\end{equation}
\end{theorem}

\section{References to Literature}

\begin{itemize}
    \item Frieden, B.R. (1998). \textit{Physics from Fisher Information}. Cambridge.
    \item Frieden, B.R. (2004). \textit{Science from Fisher Information}. Cambridge.
    \item Hall, M.J.W. (2000). Quantum mechanics from Fisher information. \textit{Phys. Rev. A}.
\end{itemize}

This is \textbf{not DD-specific} — it is established physics.

DD provides the \textbf{interpretation}: Fisher information = geometry of distinctions.
