%==============================================================================
% COSMOLOGY AND DESI OBSERVATIONS
% Empirical Validation of DD Predictions
% Part III, Chapter 7
%==============================================================================

\chapter{Cosmology and DESI Observations}\label{ch:desi}

\epigraph{The universe evolves its own laws.}{---}

\vnew{This chapter is new in DD v2.0, connecting DD to recent observations.}

\section{The Standard Cosmological Model}

The $\Lambda$CDM model assumes:
\begin{itemize}
    \item Cosmological constant $\Lambda$ is truly constant
    \item Dark energy equation of state $w = -1$ (exactly)
    \item No evolution of fundamental ``constants''
\end{itemize}

DD challenges all three assumptions.

\section{DD Prediction: Evolving Dark Energy}

\subsection{The Non-Closure Principle}

From the DD axiom $\Delta = \Delta(\Delta)$:
\begin{itemize}
    \item Reality is self-referential
    \item Self-reference implies dynamics
    \item No static ``constants'' exist
\end{itemize}

\begin{theorem}[Evolving $\Lambda$]
The cosmological ``constant'' $\Lambda(t)$ evolves in time.
\end{theorem}

\begin{proof}[Argument]
\begin{enumerate}
    \item $\Lambda$ represents the energy density of distinction space
    \item Distinction complexity $C = \text{rank}(\Delta)$ grows over cosmic time
    \item By DDCE (Chapter~\ref{ch:ddce}): $\Lambda_{\text{eff}} = k(\Delta + F + M)$
    \item As $\Delta, F, M$ increase, $\Lambda_{\text{eff}}$ increases
    \item Therefore, $\Lambda$ is not constant
\end{enumerate}
\end{proof}

\subsection{Equation of State}

The dark energy equation of state:
\begin{equation}
w = \frac{p}{\rho}
\end{equation}

For cosmological constant: $w = -1$ (exact).

DD predicts: $w \neq -1$, with time evolution.

\section{DESI 2024 Results}

\subsection{The Experiment}

The Dark Energy Spectroscopic Instrument (DESI) measures:
\begin{itemize}
    \item Baryon Acoustic Oscillations (BAO)
    \item Galaxy clustering
    \item Quasar spectra
\end{itemize}

Goal: constrain $w(z)$ as function of redshift.

\subsection{Key Results (April 2024)}

DESI Year 1 data release found:

\begin{center}
\begin{tabular}{@{}ll@{}}
\toprule
\textbf{Parameter} & \textbf{DESI + CMB + SN} \\
\midrule
$w_0$ (present value) & $-0.827 \pm 0.063$ \\
$w_a$ (evolution rate) & $-0.75^{+0.29}_{-0.25}$ \\
\bottomrule
\end{tabular}
\end{center}

\textbf{Significance}: $w_0 = -1$ excluded at $\sim 2.5\sigma$.

This is \textbf{preliminary evidence} for evolving dark energy.

\subsection{Consistency with DD}

\begin{center}
\begin{tabular}{@{}lcc@{}}
\toprule
\textbf{Prediction} & \textbf{DD} & \textbf{DESI} \\
\midrule
$w \neq -1$ & Yes & $2.5\sigma$ hint \\
$w_a \neq 0$ (evolution) & Yes & $\sim 3\sigma$ \\
$\Lambda$ not constant & Yes & Consistent \\
\bottomrule
\end{tabular}
\end{center}

\section{Theoretical Framework}

\subsection{DDCE Connection}

From Chapter~\ref{ch:ddce}:
\begin{equation}
\frac{dV}{dt} = k(\Delta + F + M)
\end{equation}

The effective cosmological term:
\begin{equation}
\Lambda_{\text{eff}}(t) = k(\Delta(t) + F(t) + M(t))
\end{equation}

This naturally gives $w \neq -1$:
\begin{equation}
w(z) = -1 + \epsilon(z)
\end{equation}

where $\epsilon(z)$ depends on the evolution of distinction complexity.

\subsection{Fisher-Ricci Interpretation}

From the master equation (Chapter~\ref{ch:info-ricci}):
\begin{equation}
\partial_t g_{ij} = -2\,\mathrm{Ric}_{ij} + 2\nabla_i\nabla_j\log p
\end{equation}

Cosmological evolution = Ricci flow of distinction geometry.

The information term $\nabla\nabla\log p$ prevents true ``constant'' $\Lambda$.

\section{Quantitative Predictions}

\subsection{Order of Magnitude}

From DDCE:
\begin{equation}
\frac{\dot{\Lambda}}{\Lambda} \sim \frac{\dot{C}}{C}
\end{equation}

where $C = \Delta + F + M$ is total distinction complexity.

Estimate: If cognitive complexity doubles over $\sim 10$ Gyr:
\begin{equation}
\frac{\dot{\Lambda}}{\Lambda} \sim \frac{\ln 2}{10 \text{ Gyr}} \sim 10^{-11} \text{ yr}^{-1}
\end{equation}

This is small but potentially detectable by precision cosmology.

\subsection{Redshift Dependence}

If $C(z) \propto (1+z)^{-\alpha}$ (complexity increases toward present):
\begin{equation}
w(z) = -1 + \alpha \frac{1}{3(1+z)^{3(1+w_0)}}
\end{equation}

This gives $w_a < 0$ (dark energy ``weakens'' in past), consistent with DESI.

\section{Future Tests}

\subsection{DESI DR2 (2025)}

Expected improvements:
\begin{itemize}
    \item $5\times$ more data
    \item Better systematics control
    \item Could reach $5\sigma$ for $w \neq -1$
\end{itemize}

\subsection{Euclid Mission (2024-2030)}

ESA's Euclid will measure:
\begin{itemize}
    \item Weak lensing
    \item Galaxy clustering to $z \sim 2$
    \item Independent test of $w(z)$
\end{itemize}

\subsection{Vera Rubin Observatory (2025+)}

LSST survey will provide:
\begin{itemize}
    \item Type Ia supernovae to $z \sim 1$
    \item Additional constraints on $w_0, w_a$
\end{itemize}

\section{Related Observational Consequences}

\subsection{Varying ``Constants''}

If $\Lambda$ evolves, other ``constants'' may too:
\begin{itemize}
    \item Fine structure constant $\alpha$
    \item Proton-to-electron mass ratio $\mu$
\end{itemize}

Current limits: $|\dot{\alpha}/\alpha| < 10^{-16}$ yr$^{-1}$.

DD predicts correlations between these variations.

\subsection{Neutrino Masses}

DESI + CMB also constrains:
\begin{equation}
\sum m_\nu < 0.072 \text{ eV} \quad (95\% \text{ CL})
\end{equation}

This is consistent with normal hierarchy, which DD may address through triadic fermion structure.

\section{Summary}

\begin{center}
\fbox{\parbox{0.9\textwidth}{
\textbf{DESI 2024 Results Support DD:}

\begin{itemize}
    \item DD predicts: $w \neq -1$, $\Lambda$ evolves
    \item DESI finds: $w_0 = -0.83 \pm 0.06$, $w_a \neq 0$ at $\sim 3\sigma$
    \item This is \textbf{not proof}, but \textbf{consistency}
    \item Future data (DESI DR2, Euclid) will test more stringently
\end{itemize}

DD is the only framework that \textbf{predicted} evolving dark energy from first principles (distinction dynamics), not fitted it post hoc.
}}
\end{center}

\section{References}

\begin{itemize}
    \item DESI Collaboration (2024). ``DESI 2024 VI: Cosmological Constraints from the Measurements of BAO.'' arXiv:2404.03002
    \item DESI Collaboration (2024). ``DESI 2024 III: Baryon Acoustic Oscillations from Galaxies and Quasars.'' arXiv:2404.03000
    \item Planck Collaboration (2020). ``Planck 2018 results. VI. Cosmological parameters.'' A\&A 641, A6.
\end{itemize}
