\chapter{Quantum Gravity from DD}

\section{The Problem of Quantum Gravity}

Two pillars of modern physics:
\begin{itemize}
\item \textbf{General Relativity}: Gravity = spacetime curvature, continuous
\item \textbf{Quantum Mechanics}: Matter = discrete quanta, probabilistic
\end{itemize}

They are incompatible. GR breaks down at Planck scale. QM doesn't include gravity consistently.

Standard approaches (string theory, loop quantum gravity) add structure. DD derives both from one principle.

\section{DD Resolution: Both Are Distinction Dynamics}

\subsection{Quantum Mechanics from DD}

Following Frieden's extreme physical information:
\[
I_F = \int p(x) \left( \frac{\partial \log p}{\partial x} \right)^2 dx
\]

Minimizing Fisher information with constraints yields Schrödinger equation:
\[
i\hbar \frac{\partial \psi}{\partial t} = -\frac{\hbar^2}{2m} \nabla^2 \psi + V\psi
\]

DD interpretation: $\psi$ encodes \emph{distinguishability structure} of system.

\subsection{General Relativity from DD}

Ricci flow:
\[
\frac{\partial g_{ij}}{\partial t} = -2 R_{ij}
\]

Fixed point (Einstein equations):
\[
R_{\mu\nu} - \frac{1}{2} g_{\mu\nu} R = 8\pi G T_{\mu\nu}
\]

DD interpretation: $g_{\mu\nu}$ encodes \emph{geometric structure} of distinction space.

\subsection{The Unification}

\begin{center}
\fbox{\parbox{0.8\textwidth}{
\textbf{Key Insight}:

Fisher metric on probability space = Ricci curvature on manifold

Both measure the same thing: \emph{distinguishability geometry}

QM and GR are two aspects of $\Delta$-dynamics.
}}
\end{center}

\section{The Planck Scale}

\subsection{Why Planck Units?}

Planck length: $\ell_P = \sqrt{\frac{\hbar G}{c^3}} \approx 10^{-35}$ m

In DD: this is where distinction becomes \emph{indistinguishable from itself}.

\[
\ell_P = \text{minimum distinguishable length}
\]

Below $\ell_P$, the concept of ``two points'' loses meaning. Not because of measurement limits, but because distinction itself has minimum resolution.

\subsection{Planck Scale as Triadic Limit}

The triad requires three distinguishable elements. At Planck scale:
\begin{itemize}
\item Only one distinction fits
\item Triad cannot form
\item Structure simplifies to primitive $\Delta$
\end{itemize}

This is why physics ``breaks down'' at Planck scale: triadic complexity (which gives rise to space, gauge groups, etc.) cannot exist.

\section{Spacetime Emergence}

\subsection{Space from Distinctions}

Standard view: space is given, things exist in it.

DD view: things (distinctions) exist, space emerges from their relations.

\[
d(A, B) = f(\Delta(A, B))
\]

Distance is a \emph{function} of distinguishability, not the other way around.

\subsection{Time from Distinction Ordering}

Standard view: time is given, events occur in it.

DD view: events (distinction acts) occur, time emerges as their ordering.

\[
t_A < t_B \iff \Delta_A \text{ precedes } \Delta_B \text{ in causal chain}
\]

\subsection{Emergent Lorentz Invariance}

Why is spacetime Lorentzian (signature $-+++$)?

DD answer: Time (ordering of $\Delta$) is categorically different from space (relations between $\Delta$). This asymmetry gives signature.

The speed of light $c$ = maximum rate at which distinctions can propagate = maximum ``update speed'' of $\Delta$-field.

\section{Quantum Gravity Equations}

\subsection{DD-Modified Einstein Equations}

\[
R_{\mu\nu} - \frac{1}{2} g_{\mu\nu} R + \Lambda_\Delta g_{\mu\nu} = 8\pi G \left( T_{\mu\nu} + T^{(\Delta)}_{\mu\nu} \right)
\]

where:
\begin{align}
\Lambda_\Delta &= \lambda \frac{d|\Delta|}{dt} \quad \text{(distinction generation)} \\
T^{(\Delta)}_{\mu\nu} &= \text{stress-energy of distinction field}
\end{align}

\subsection{DD-Modified Schrödinger Equation}

\[
i\hbar \frac{\partial \psi}{\partial t} = \hat{H} \psi + \gamma R \psi
\]

where $R$ = Ricci scalar, $\gamma$ = coupling constant.

Gravity affects quantum evolution through curvature term.

\subsection{Self-Consistency}

The equations must be self-consistent:
\begin{itemize}
\item $\psi$ determines $T_{\mu\nu}$
\item $T_{\mu\nu}$ determines $g_{\mu\nu}$
\item $g_{\mu\nu}$ (through $R$) affects $\psi$
\end{itemize}

Closed loop. This is $\Delta(\Delta)$ at the level of equations.

\section{Resolution of Singularities}

\subsection{Black Hole Singularity}

Classical GR: $r \to 0$, curvature $\to \infty$

DD: At $r \sim \ell_P$, distinction structure changes. ``Inside'' and ``outside'' lose meaning. No singularity---just transition to primitive $\Delta$ state.

\subsection{Big Bang Singularity}

Classical GR: $t \to 0$, density $\to \infty$

DD: At $t \sim t_P$, time itself (as ordering of $\Delta$) loses meaning. No ``before'' the Big Bang---not because we can't know, but because the concept doesn't apply.

\subsection{Information Paradox}

Hawking: Information lost in black holes?

DD: Information = distinctions. Distinctions cannot be destroyed (only transformed). Black hole evaporation must preserve distinction structure, perhaps in correlations of Hawking radiation.

\section{Predictions}

\begin{enumerate}
\item \textbf{Minimum length}: Experiments probing $\ell_P$ should see discreteness effects
\item \textbf{Modified dispersion}: $E^2 = p^2 c^2 + m^2 c^4 + \alpha \frac{E^3}{E_P}$
\item \textbf{Decoherence from gravity}: Superpositions of massive objects decohere due to gravitational self-interaction
\item \textbf{Cosmological signatures}: Primordial gravitational waves should show discrete spectrum
\end{enumerate}

\section{Comparison with Other Approaches}

\begin{center}
\begin{tabular}{l|c|c|c}
& \textbf{String Theory} & \textbf{LQG} & \textbf{DD} \\
\hline
Extra dimensions & 6-7 & 0 & 0 \\
New entities & Strings & Spin networks & None \\
Derived from & Consistency & Quantization & Single axiom \\
Predicts SM & Landscape & No & Yes \\
Testable & Difficult & Some & Yes \\
\end{tabular}
\end{center}

\section{Open Questions}

\begin{itemize}
\item Exact form of $T^{(\Delta)}_{\mu\nu}$?
\item Value of coupling $\gamma$?
\item Detailed black hole interior structure?
\item Connection to holographic principle?
\end{itemize}

\section{Summary}

Quantum gravity in DD is not a \emph{unification} of two theories. It's recognition that both theories are \emph{already} aspects of one phenomenon: distinction dynamics.

\begin{center}
\begin{tabular}{c|c}
\textbf{Quantum Mechanics} & \textbf{General Relativity} \\
\hline
Distinguishability of states & Distinguishability of events \\
Fisher information & Ricci curvature \\
$\hbar$ & $G$ \\
Superposition & Geometry
\end{tabular}
\end{center}

Both describe how distinctions behave. The ``problem'' of quantum gravity is an artifact of treating them as separate.
