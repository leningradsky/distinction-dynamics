%==============================================================================
% PHYSICAL CONSTANTS FROM DD
% Constraints and Relations
% Part III, Chapter 6
%==============================================================================

\chapter{Physical Constants}\label{ch:constants}

\epigraph{The constants are not chosen. They are computed — or they are impossible.}{---}

\vnew{This chapter explores what DD can and cannot say about constants.}

\section{The Problem of Constants}

The Standard Model has $\sim 19$ free parameters:
\begin{itemize}
    \item 3 gauge couplings: $g_1, g_2, g_3$
    \item 9 Yukawa couplings (fermion masses)
    \item 4 CKM parameters
    \item Higgs parameters: $\mu, \lambda$
    \item $\theta_{QCD}$
\end{itemize}

Plus gravitational constant $G$ and cosmological constant $\Lambda$.

\textbf{Can DD derive any of these?}

\section{What DD Claims}

From Chapter~\ref{ch:dd-complete} (DD-Completeness):

\begin{theorem}[Zero Free Parameters]
In the final theory, all parameters are fixed by self-consistency. There are no genuinely free choices.
\end{theorem}

However, DD in its current form does not compute specific values.

What DD \textbf{does} provide:
\begin{enumerate}
    \item Relations between constants
    \item Constraints from consistency
    \item Qualitative predictions
\end{enumerate}

\section{Coupling Constant Unification}

\subsection{Running Couplings}

The gauge couplings run with energy:
\begin{equation}
\frac{1}{\alpha_i(\mu)} = \frac{1}{\alpha_i(M_Z)} - \frac{b_i}{2\pi} \ln\frac{\mu}{M_Z}
\end{equation}

\subsection{DD Prediction}

\begin{theorem}[Coupling Relation]
At some scale $M_U$, all gauge couplings should unify:
\begin{equation}
\alpha_1(M_U) = \alpha_2(M_U) = \alpha_3(M_U)
\end{equation}
\end{theorem}

\begin{proof}[Argument from DD]
\begin{enumerate}
    \item At high energy, all distinctions are equivalent
    \item Distinct gauge groups emerge from one structure
    \item Therefore, couplings should match at unification scale
\end{enumerate}
\end{proof}

\textbf{Observation}: In the SM, couplings nearly unify at $M_U \sim 10^{16}$ GeV.

With supersymmetry, unification is exact.

\section{The Fine Structure Constant}

\subsection{The Value}

\begin{equation}
\alpha = \frac{e^2}{4\pi\epsilon_0 \hbar c} \approx \frac{1}{137.036}
\end{equation}

\subsection{DD Approach}

We cannot derive $\alpha = 1/137$ from first principles yet.

However, we can note:

\begin{theorem}[Eddington-DD Relation]
$\alpha^{-1} \approx 137$ may be related to triadic combinatorics.
\end{theorem}

Speculation (not proven):
\begin{itemize}
    \item $137 = 128 + 8 + 1 = 2^7 + 2^3 + 2^0$
    \item Or related to representation dimensions
\end{itemize}

This remains an \textbf{open problem}.

\section{Mass Ratios}

\subsection{Fermion Mass Hierarchy}

\begin{equation}
\frac{m_t}{m_e} \sim 3.4 \times 10^5
\end{equation}

This enormous hierarchy is unexplained in SM.

\subsection{DD Perspective}

\begin{theorem}[Mass from Distinction Strength]
Fermion mass $\propto$ coupling to Higgs $\propto$ distinction strength with vacuum.
\end{theorem}

Different generations have different ``positions'' in triadic structure:
\begin{itemize}
    \item 3rd generation: closest to triadic center (strongest coupling)
    \item 1st generation: furthest from center (weakest coupling)
\end{itemize}

This is qualitative, not quantitative.

\section{Planck Units}

\subsection{Natural Units}

\begin{align}
l_P &= \sqrt{\frac{\hbar G}{c^3}} \approx 1.6 \times 10^{-35} \text{ m} \\
t_P &= \sqrt{\frac{\hbar G}{c^5}} \approx 5.4 \times 10^{-44} \text{ s} \\
m_P &= \sqrt{\frac{\hbar c}{G}} \approx 2.2 \times 10^{-8} \text{ kg}
\end{align}

\subsection{DD Interpretation}

\begin{theorem}[Planck Scale as Distinction Limit]
The Planck scale is where distinction itself becomes quantum:
\begin{equation}
\Delta x \cdot \Delta p \geq \frac{\hbar}{2}
\end{equation}
At $l_P$, the uncertainty in distinction equals the distinction itself.
\end{theorem}

Therefore: Planck scale = \textbf{resolution limit of distinction}.

\section{The Cosmological Constant Problem}

\subsection{The Puzzle}

Observed: $\Lambda \sim 10^{-122} M_P^4$

QFT prediction: $\Lambda \sim M_P^4$

Discrepancy: 122 orders of magnitude!

\subsection{DD Resolution (DDCE)}

From Chapter~\ref{ch:ddce}:

\begin{equation}
\Lambda_{\text{eff}} = k(\Delta + F + M)
\end{equation}

The cosmological ``constant'' is \textbf{not constant} — it's determined by the current state of distinction complexity.

\begin{theorem}[Dynamic $\Lambda$]
The small observed $\Lambda$ reflects the current cosmic epoch's distinction structure, not a fundamental constant.
\end{theorem}

This connects to DESI observations of evolving dark energy.

\section{The Hierarchy Problem}

\subsection{The Puzzle}

Why is the Higgs mass ($\sim 125$ GeV) so much smaller than the Planck scale ($\sim 10^{19}$ GeV)?

\subsection{DD Perspective}

\begin{theorem}[Hierarchy from Information Levels]
The hierarchy reflects the ratio between:
\begin{itemize}
    \item Fundamental distinction scale (Planck)
    \item Electroweak distinction scale (Higgs)
\end{itemize}
\end{theorem}

The Higgs scale is where \textbf{matter distinctions crystallize}.

This is lower than the \textbf{spacetime distinction scale} (Planck).

Full derivation remains open.

\section{What DD Predicts}

\subsection{Qualitative}

\begin{enumerate}
    \item Coupling unification at high energy
    \item Mass hierarchy correlates with generation number
    \item $\Lambda$ evolves with cosmic time
    \item Planck scale = distinction resolution limit
\end{enumerate}

\subsection{Testable Relations}

\begin{theorem}[DDCE Prediction]
\begin{equation}
w(z) \neq -1
\end{equation}
The dark energy equation of state evolves with redshift.
\end{theorem}

This is being tested by DESI, Euclid, and future surveys.

\section{What DD Does Not (Yet) Derive}

\begin{itemize}
    \item $\alpha = 1/137.036...$
    \item $m_e/m_p = 1/1836...$
    \item Specific Yukawa couplings
    \item CKM matrix elements
    \item $\theta_{QCD} \approx 0$
\end{itemize}

These may require:
\begin{itemize}
    \item Deeper understanding of triadic number theory
    \item Explicit computation on Fisher-Ricci manifold
    \item Connection to string/M-theory landscape
\end{itemize}

\section{Summary}

\begin{center}
\fbox{\parbox{0.9\textwidth}{
\textbf{Status of Constants in DD:}

\begin{itemize}
    \item DD claims all constants are \textbf{computable in principle}
    \item Currently: qualitative relations and constraints
    \item Key prediction: $\Lambda$ evolves (testable via DESI)
    \item Open problem: derive specific numerical values
\end{itemize}

The constants are not free parameters. They are fixed by self-consistency. We just haven't computed them all yet.
}}
\end{center}
