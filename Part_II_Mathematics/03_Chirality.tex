%==============================================================================
% CHIRALITY
% The Handedness of Self-Reference
% Part II, Chapter 3
%==============================================================================

\chapter{Chirality}\label{ch:chirality}

\epigraph{Self-reference has a direction.}{---}

\section{The Origin of Handedness}

\begin{theorem}[Chirality from Self-Application]
The self-application $\Delta(\Delta)$ has an inherent direction.
\end{theorem}

\begin{proof}
In $\Delta(\Delta)$:
\begin{itemize}
    \item The outer $\Delta$ is the operator
    \item The inner $\Delta$ is the operand
\end{itemize}

This is not symmetric:
\begin{equation}
\text{outer} \to \text{inner} \neq \text{inner} \to \text{outer}
\end{equation}

The direction of application defines chirality.
\end{proof}

\section{Mathematical Realization}

\begin{theorem}[Chirality in Lie Algebras]
The Lie bracket $[X, Y] = XY - YX$ encodes chirality.
\end{theorem}

\begin{proof}
\begin{itemize}
    \item $[X, Y] = -[Y, X]$ (antisymmetry)
    \item Switching order reverses sign
    \item This is the algebraic manifestation of chirality
\end{itemize}
\end{proof}

\section{Physical Manifestations}

\subsection{Weak Interactions}

\begin{theorem}[Parity Violation]
The weak force distinguishes left from right because distinction itself does.
\end{theorem}

\begin{itemize}
    \item Left-handed fermions couple to $W^\pm$
    \item Right-handed fermions do not
    \item This is the $SU(2)_L$ structure
\end{itemize}

\subsection{CP Violation}

\begin{theorem}[CP Violation from Triadic Phase]
CP violation arises from the non-eliminable phase in triadic structures.
\end{theorem}

In the CKM matrix:
\begin{equation}
V_{CKM} = \begin{pmatrix} V_{ud} & V_{us} & V_{ub} \\ V_{cd} & V_{cs} & V_{cb} \\ V_{td} & V_{ts} & V_{tb} \end{pmatrix}
\end{equation}

The complex phases cannot all be removed by rephasing. This asymmetry is related to the asymmetry of $\Delta(\Delta)$.

\subsection{Biological Chirality}

\begin{theorem}[Life is Chiral]
Living systems use only one handedness of molecules.
\end{theorem}

\begin{itemize}
    \item Amino acids: L (left-handed)
    \item Sugars: D (right-handed)
\end{itemize}

This may trace back to parity violation in weak interactions, which is itself from DD chirality.

\section{Chirality and Time}

\begin{theorem}[Time Direction from Chirality]
The arrow of time is related to the chirality of self-reference.
\end{theorem}

\begin{proof}[Argument]
\begin{itemize}
    \item $\Delta(\Delta)$: outer applies to inner
    \item This defines a sequence: first outer, then inner
    \item Sequence is the primitive form of time ordering
    \item The direction of the sequence is the arrow of time
\end{itemize}
\end{proof}

\section{Parity as Discrete Symmetry}

\begin{definition}[Parity Transformation]
Parity $P$ reverses the direction of application:
\begin{equation}
P: \Delta(\Delta) \to \Delta^{-1}(\Delta^{-1})
\end{equation}
\end{definition}

In physics:
\begin{equation}
P: \vec{x} \to -\vec{x}, \quad P: \vec{p} \to -\vec{p}
\end{equation}

\section{Chirality and Complexity}

\begin{theorem}[Chiral Systems are More Complex]
Breaking of parity symmetry increases distinction capacity.
\end{theorem}

\begin{proof}
\begin{itemize}
    \item Parity-symmetric system: L = R
    \item Parity-broken system: L $\neq$ R
    \item Breaking adds a new distinction (which hand)
    \item More distinctions = more complexity
\end{itemize}
\end{proof}

\section{Summary}

\begin{center}
\fbox{\parbox{0.85\textwidth}{
\textbf{Result}: Chirality is intrinsic to self-reference.

\begin{itemize}
    \item $\Delta(\Delta)$ has a direction (outer $\to$ inner)
    \item This manifests as parity violation in physics
    \item CP violation comes from triadic phase
    \item Biological homochirality may trace to weak force
    \item Time direction is related to application direction
\end{itemize}
}}
\end{center}
