%==============================================================================
% Chapter: Complexity as Rank
% Part II: Mathematics
% NEW IN v2.0
%==============================================================================

\chapter{Complexity as Rank}\label{ch:complexity}

\epigraph{Complexity is not quantity, but dimension.}{---}

\vnew{This chapter is new in DD v2.0.}

\section{The Definition}

\begin{definition}[Complexity]\label{def:complexity-full}
The \textbf{complexity} of a distinction system $\Delta$ is:
\[
\Complexity = \rank(\Delta)
\]
where $\rank$ is the number of independent axes of distinction.
\end{definition}

\begin{remark}
This is not the number of elements, but the dimension of the distinction space.
\end{remark}

\section{Properties}

\begin{theorem}[Complexity Hierarchy]
\begin{itemize}
    \item Monad: $\Complexity = 0$ (no distinction)
    \item Dyad: $\Complexity = 1$ (one axis)
    \item Triad: $\Complexity = 2$ (two independent axes)
    \item $n$-ad: $\Complexity \leq n-1$
\end{itemize}
\end{theorem}

\begin{theorem}[Minimal Complexity for World]
$\Complexity \geq 2$ is required for a self-sufficient ontological structure.
\end{theorem}

\begin{proof}
By Chapter~\ref{ch:dyad}, dyad ($\Complexity = 1$) is insufficient. Triad ($\Complexity = 2$) is minimal.
\end{proof}

\section{Consequences for Intelligence}

\begin{corollary}[AI Limitation]
Current AI systems have fixed $\rank(\Delta)$. They increase data within fixed complexity, not complexity itself. True intelligence expands $\rank(\Delta)$.
\end{corollary}

\begin{corollary}[Human Uniqueness]
Humans can distinguish distinctions recursively: $\Delta^{(n)}$ for arbitrary $n$. This gives effectively unbounded $\Complexity$.
\end{corollary}

% TODO: Formal proofs, connection to information theory
