%==============================================================================
% P ≠ NP
% A Structural Approach via DD
% Part II, Chapter 5
%==============================================================================

\chapter{P $\neq$ NP}\label{ch:pnp}

\epigraph{The triad is the minimum closed structure. There computational hardness begins.}{---}

\section{The Problem}

\begin{definition}[P vs NP]
\begin{itemize}
    \item \textbf{P}: Problems solvable in polynomial time
    \item \textbf{NP}: Problems verifiable in polynomial time
\end{itemize}
\end{definition}

\textbf{Question}: Is P = NP?

\textbf{DD Claim}: P $\neq$ NP follows from the irreducibility of the triad to the dyad.

\section{Empirical Pattern}

A striking pattern in complexity theory:

\begin{center}
\begin{tabular}{|l|c|c|c|}
\hline
\textbf{Problem} & \textbf{Constraint} & \textbf{Domain} & \textbf{Class} \\
\hline
2-SAT & 2 & 2 & P \\
3-SAT & 3 & 2 & NP-complete \\
2-COLORING & 2 & 2 & P \\
3-COLORING & 2 & 3 & NP-complete \\
HORN-SAT & $\geq 3$ & 2 & P \\
\hline
\end{tabular}
\end{center}

\textbf{Observation}: NP-completeness emerges when ``3'' appears—either in constraint scope or in domain size.

\textbf{Exception}: HORN-SAT has clauses of size $\geq 3$ but is in P. Why?

\section{DD Interpretation: Closed vs Open Triads}

\begin{definition}[Closed Triad]
A constraint $C(x,y,z)$ is a \emph{closed triad} if:
\begin{enumerate}
    \item $|\text{scope}(C)| \geq 3$
    \item $C$ is symmetric under permutation of variables
    \item $C$ does not factorize: $C \not\equiv C_1(x,y) \land C_2(y,z)$
\end{enumerate}
\end{definition}

\begin{definition}[Open (Directed) Triad]
A constraint $C$ is an \emph{open triad} if it has an inherent direction:
\begin{equation}
(x_1 \land x_2 \land \ldots) \to y
\end{equation}
One variable is distinguished as the ``output.''
\end{definition}

\begin{theorem}[Classification]
\begin{itemize}
    \item 3-SAT clause $(x \lor y \lor z)$: symmetric, closed triad
    \item HORN clause $(\neg x \lor \neg y \lor z)$: directed, open triad
\end{itemize}
\end{theorem}

\textbf{DD Principle}: Closed triads are irreducible. Open triads allow sequential resolution.



\section{The Branching Factor}

\begin{definition}[Branching Factor]
For a constraint $C$ and partial assignment $\sigma$:
\begin{equation}
\text{branch}(C, \sigma) = |\{\tau : \tau \text{ extends } \sigma \text{ and satisfies } C\}|
\end{equation}
\end{definition}

\begin{theorem}[Branching Characterization]
\begin{itemize}
    \item 2-SAT: $\text{branch}(C, \sigma) \leq 1$ for any unit propagation
    \item HORN-SAT: $\text{branch}(C, \sigma) \leq 1$ for any unit propagation
    \item 3-SAT: $\text{branch}(C, \sigma) = 2$ after one falsified literal
    \item 3-COLORING: $\text{branch}(C, \sigma) = 2$ after one colored vertex
\end{itemize}
\end{theorem}

\textbf{Pattern}: 
\begin{equation}
\text{branch} \leq 1 \implies \text{P}, \quad \text{branch} \geq 2 \implies \text{potentially NP-complete}
\end{equation}

\section{Resolution Cascade}

\begin{definition}[Cascade Width]
The \emph{cascade width} of a resolution process is the maximum number of simultaneous unresolved choices at any step.
\end{definition}

\begin{theorem}[P Characterization]
A CSP is in P if every resolution strategy has cascade width $O(1)$.
\end{theorem}

\begin{proof}[Argument]
\begin{itemize}
    \item Width $O(1)$: at most constant choices at any time
    \item Total paths: $O(1)^n = O(1)$ (polynomial)
    \item Deterministic traversal suffices
\end{itemize}
\end{proof}

\begin{theorem}[NP Characterization]  
A CSP is NP-complete if minimum cascade width is $\Omega(n)$.
\end{theorem}

\begin{proof}[Argument]
\begin{itemize}
    \item Width $\Omega(n)$: linear choices accumulate
    \item Total paths: $2^{\Omega(n)}$ (exponential)
    \item Must traverse all paths in worst case
\end{itemize}
\end{proof}

\section{Connection to Tree-Width}

\begin{definition}[Hypertree-Width]
For a CSP with constraint hypergraph $H$:
\begin{equation}
\text{htw}(H) = \text{minimum width of hypertree decomposition}
\end{equation}
\end{definition}

\begin{theorem}[Gottlob et al.]
CSP with $\text{htw} \leq k$ is solvable in time $O(n^{k+1} \cdot |D|^{k+1})$.
\end{theorem}

\textbf{Interpretation}: Tree-width measures how ``tree-like'' the constraint structure is. Trees are sequences of binary choices—dyadic. High tree-width means triadic entanglement.



\section{What We Tried and Rejected}

Intellectual honesty requires documenting failed approaches.

\subsection{Rejected: Triad = Size 3}

\textbf{Hypothesis}: Any constraint with 3+ variables creates NP-hardness.

\textbf{Counterexample}: HORN-SAT has clauses of size $\geq 3$ but is in P.

\textbf{Lesson}: It's not the size, it's the \emph{structure}. HORN clauses are directed (open triads), not symmetric (closed triads).

\subsection{Rejected: Closedness via Automorphisms}

\textbf{Hypothesis}: Closed = transitive automorphism group.

\textbf{Problem}: Both $\{a,b\}$ and $\{a,b,c\}$ with symmetric relations have transitive automorphism groups. Cannot distinguish dyad from triad this way.

\subsection{Rejected: Full Equivalence}

\textbf{Hypothesis}: P $\neq$ NP $\iff$ no polynomial reduction 3-SAT $\to$ 2-SAT.

\textbf{Problem}: Only one direction holds:
\begin{align}
\exists \text{ poly reduction 3-SAT} \to \text{2-SAT} &\implies \text{P} = \text{NP} \quad \checkmark \\
\text{P} \neq \text{NP} &\implies \nexists \text{ such reduction} \quad \checkmark \\
\nexists \text{ such reduction} &\implies \text{P} \neq \text{NP} \quad \text{?}
\end{align}

The last implication is not proven.

\subsection{Rejected: htw Characterizes the Problem}

\textbf{Hypothesis}: 3-SAT has $\text{htw} = \Theta(n)$.

\textbf{Problem}: Hypertree-width characterizes \emph{instances}, not problems. Some 3-SAT instances have low htw and are easy.

\textbf{Correct statement}: There exist 3-SAT instances with $\text{htw} = \Theta(n)$.

\subsection{Rejected: Unconditional Lower Bound via htw}

\textbf{Hypothesis}: High htw implies exponential time.

\textbf{Problem}: Known results (Marx 2010) are conditional on ETH. The Exponential Time Hypothesis itself is unproven.

\textbf{Circular dependency}: We wanted to prove ETH, not assume it.



\section{What Survives}

After systematic attack, these claims remain standing:

\begin{enumerate}
    \item \textbf{Triadic threshold}: NP-completeness emerges at exactly 3 (scope or domain)
    \item \textbf{Closed vs open}: Symmetric (closed) triads correlate with hardness; directed (open) triads allow polynomial algorithms
    \item \textbf{Branching characterization}: $\text{branch} \leq 1 \to$ P; $\text{branch} \geq 2 \to$ potential NP-completeness
    \item \textbf{Cascade width}: Polynomial solvability requires bounded cascade width
    \item \textbf{Partial formalization}: Tree-width/hypertree-width capture some of this structure
\end{enumerate}

\section{The DD Argument}

\begin{theorem}[Why 3 Is Special]
From DD core:
\begin{itemize}
    \item 1 = identity (no distinction)
    \item 2 = dyad (distinction without position—degenerate)
    \item 3 = triad (minimum closed structure)
\end{itemize}
\end{theorem}

\begin{proof}
To distinguish $a$ from $b$, we need position $c$. The triple $\{a,b,c\}$ is the minimum where each element has a position relative to two others. This closes the structure—there is no ``outside.''
\end{proof}

\begin{theorem}[Computational Consequence]
Closed structures require internal resolution. Open structures allow external leverage.
\begin{itemize}
    \item \textbf{External}: Algorithm stands outside, applies distinctions sequentially
    \item \textbf{Internal}: Algorithm must enter the structure, choices cascade
\end{itemize}
\end{theorem}

\begin{corollary}[P vs NP Interpretation]
\begin{align}
\text{P} &= \text{problems with external resolution (dyadic)} \\
\text{NP} &= \text{problems requiring internal resolution (triadic)}
\end{align}
\end{corollary}

\section{The Gap}

DD provides:
\begin{itemize}
    \item \textbf{Explanation}: Why 3 is the threshold
    \item \textbf{Characterization}: Closed vs open triads
    \item \textbf{Correlation}: Triadic structure $\leftrightarrow$ NP-completeness
\end{itemize}

DD does not provide:
\begin{itemize}
    \item \textbf{Proof}: That P $\neq$ NP
    \item \textbf{Unconditional lower bound}: Independent of ETH
    \item \textbf{Barrier avoidance}: Formal demonstration
\end{itemize}

\section{Meta-Observation}

\begin{theorem}[Why Mathematical Proof May Be Elusive]
P $\neq$ NP is a statement about the structure of distinction itself. Mathematics \emph{describes} this structure but \emph{arises from} it. 

Proving P $\neq$ NP from within mathematics may face the same limitation as proving consistency from within arithmetic (Gödel).
\end{theorem}

\textbf{Prediction}: Any future proof of P $\neq$ NP will explicitly or implicitly use the irreducibility of triadic structures to dyadic ones.



\section{Summary}

\begin{center}
\fbox{\parbox{0.9\textwidth}{
\textbf{Central Claim}: P $\neq$ NP is the complexity-theoretic expression of triad irreducibility.

\vspace{0.5em}
\textbf{Evidence}:
\begin{itemize}
    \item 2-SAT $\in$ P, 3-SAT NP-complete
    \item 2-COLORING $\in$ P, 3-COLORING NP-complete  
    \item HORN-SAT $\in$ P despite large clauses (directed $\to$ open triad)
    \item Branching factor: 1 $\to$ P, $\geq$2 $\to$ NP-hard
\end{itemize}

\vspace{0.5em}
\textbf{DD Explanation}:
\begin{itemize}
    \item Triad = minimum closed structure
    \item Closed = no external point of leverage
    \item Internal resolution = exponential cascade
\end{itemize}

\vspace{0.5em}
\textbf{Formal Connection}:
\begin{itemize}
    \item Tree-width measures ``triadicity''
    \item Low tree-width $\to$ polynomial (dyadic decomposition exists)
    \item High tree-width $\to$ exponential (under ETH)
\end{itemize}

\vspace{0.5em}
\textbf{Status}: 
\begin{itemize}
    \item Explanation: \textbf{Complete}
    \item Characterization: \textbf{Partial}
    \item Proof: \textbf{Open}
\end{itemize}
}}
\end{center}

\section{Open Questions}

\begin{enumerate}
    \item Can ``closed triad'' be formalized without reference to symmetry?
    \item Is there a static (non-dynamic) characterization that distinguishes dyad from triad computationally?
    \item Can the connection between htw and time complexity be made unconditional?
    \item Does $\Delta(\Delta)$ (meta-distinction) have a complexity-theoretic interpretation beyond nondeterminism?
\end{enumerate}

\section{Methodological Note}

This chapter documents not just results but the \emph{process} of elimination. 

\begin{quote}
Every thought is valuable: if not true, it eliminates from what remains.
\end{quote}

Failed approaches are recorded because:
\begin{itemize}
    \item They prevent others from repeating the same paths
    \item They delineate the boundary of what works
    \item They often contain partial insights that inform successful approaches
\end{itemize}

The search for P $\neq$ NP via DD continues.
