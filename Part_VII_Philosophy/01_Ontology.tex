\chapter{DD Ontology: What Exists}

\section{The Fundamental Question}

Philosophy begins with: What exists?

Traditional answers:
\begin{itemize}
\item \textbf{Materialism}: Matter exists, everything else emerges
\item \textbf{Idealism}: Mind exists, matter is appearance
\item \textbf{Dualism}: Both exist independently
\item \textbf{Neutral monism}: Something underlies both
\end{itemize}

DD answer: \textbf{Distinction exists}. Matter and mind are aspects of $\Delta$.

\section{Ontological Primitives}

\subsection{What Is Most Basic?}

\begin{center}
\begin{tabular}{l|l}
\textbf{Framework} & \textbf{Primitive} \\
\hline
Physics & Particles/fields \\
Information theory & Bits \\
Process philosophy & Events \\
DD & Distinctions \\
\end{tabular}
\end{center}

\subsection{Why Distinction?}

Particles presuppose distinction (this particle vs. that)

Bits presuppose distinction (0 vs. 1)

Events presuppose distinction (before vs. after)

\textbf{Distinction is presupposed by all frameworks.}

\section{The Self-Grounding}

\subsection{The Problem of Foundation}

Every foundation needs justification. Infinite regress?

\subsection{DD Solution}

$\Delta$ grounds itself:
\begin{itemize}
\item To deny $\Delta$ requires making a distinction
\item ``No distinction'' is itself a distinction
\item $\Delta \neq \varnothing$ is self-confirming
\end{itemize}

This is not circular but \emph{self-grounding}. Like the Cogito, but more fundamental.

\section{Categories of Being}

\subsection{Traditional Categories}

Aristotle: Substance, quality, quantity, relation, etc.

Kant: Unity, plurality, totality; reality, negation, limitation; etc.

\subsection{DD Categories}

From $\Delta$, we derive:

\begin{enumerate}
\item \textbf{Identity}: $A = A$ (self-distinction)
\item \textbf{Difference}: $A \neq B$ (distinction proper)
\item \textbf{Relation}: $A - B$ (distinction structure)
\item \textbf{Unity}: $A \cup B$ (collapsed distinction)
\item \textbf{Process}: $A \to B$ (distinction change)
\end{enumerate}

All other categories reduce to these.

\section{Existence and Non-Existence}

\subsection{What Does ``Exist'' Mean?}

To exist = to be distinguishable.

\[
\text{Exists}(X) \iff \exists Y: X \neq Y
\]

\subsection{Can Non-Existence Exist?}

``Nothing'' cannot exist as object, because ``nothing'' = no distinctions = no existence.

But ``nothing'' as \emph{concept} exists (we can think about it).

The concept of nothing is a distinction. The referent of ``nothing'' is not.

\section{Universals and Particulars}

\subsection{The Problem}

Do abstract entities (numbers, properties) exist?

Platonism: Yes, in a separate realm.

Nominalism: No, only names.

\subsection{DD Solution}

Universals = \emph{invariants} of distinction structures.

They ``exist'' not as objects but as \emph{patterns that recur}.

\begin{center}
\fbox{\parbox{0.8\textwidth}{
``Red'' exists as the invariant across all red things.

Not in a Platonic heaven, but in the structure of distinctions.
}}
\end{center}

\section{Substance and Process}

\subsection{Traditional View}

Substances persist through time. Properties change.

\subsection{DD View}

There are no substances. Only processes (distinction acts).

What appears as ``substance'' = stable pattern of distinctions.

\[
\text{``Electron''} = \text{stable } \Delta \text{-pattern with certain invariants}
\]

\section{Space and Time}

\subsection{Are They Real?}

Newton: Absolute space and time exist.

Leibniz: Space and time are relations.

\subsection{DD Answer}

Space = relation between coexisting distinctions.

Time = ordering of distinction acts.

Neither is absolute. Both emerge from $\Delta$.

\section{Causation}

\subsection{What Is Causation?}

Hume: Regular succession (no necessary connection).

Kant: Category of understanding.

\subsection{DD View}

Causation = distinction generating distinction.

\[
A \xrightarrow{\text{causes}} B \iff \Delta(A) \to \Delta(B)
\]

Necessity comes from the structure of $\Delta$, not from observation.

\section{Modality}

\subsection{Possible and Necessary}

What could have been otherwise? What must be as it is?

\subsection{DD View}

\textbf{Necessary}: What follows from $\Delta \neq \varnothing$.

\textbf{Contingent}: What depends on which $\Delta$ are made.

Physical laws = necessary (from triadic structure).

Initial conditions = contingent (particular $\Delta$ configuration).

\section{The Mind-Body Problem}

\subsection{Traditional Formulation}

How does physical brain relate to mental experience?

\subsection{DD Dissolution}

The problem assumes matter and mind are different kinds.

DD: Both are $\Delta$. There's no gap to bridge.

\begin{align}
\text{Brain state} &= \text{physical } \Delta \text{-pattern} \\
\text{Mental state} &= \text{self-referential } \Delta \text{-pattern}
\end{align}

Same stuff, different organization.

\section{Personal Identity}

\subsection{What Makes Me Me?}

Same body? Changes completely over years.

Same memories? Memories can be false.

Same soul? What is a soul?

\subsection{DD View}

Personal identity = persistent pattern of self-referential distinctions.

``I'' = $\Delta(\Delta)$ that maintains continuity.

You are not a thing. You are a process---a self-maintaining distinction structure.

\section{Free Will}

\subsection{The Dilemma}

If determinism: No freedom (everything caused).

If indeterminism: No control (random is not free).

\subsection{DD View}

$\Delta(\Delta)$ is neither determined nor random.

It is \emph{self-caused}. The distinction distinguishes itself.

Freedom = being the origin of your own distinctions.

\section{Summary: DD Ontology}

\begin{center}
\begin{tabular}{l|l}
\textbf{Question} & \textbf{DD Answer} \\
\hline
What exists? & Distinctions \\
Why something? & $\Delta \neq \varnothing$ self-confirming \\
Universals? & Invariant patterns \\
Substances? & Stable processes \\
Causation? & $\Delta$ generating $\Delta$ \\
Mind-body? & Same $\Delta$, different structure \\
Free will? & Self-caused $\Delta(\Delta)$ \\
\end{tabular}
\end{center}

\begin{center}
\fbox{\parbox{0.8\textwidth}{
\textbf{DD Ontology in one sentence:}

To be is to be distinguished.

$\text{Esse} = \text{Distingui}$
}}
\end{center}
