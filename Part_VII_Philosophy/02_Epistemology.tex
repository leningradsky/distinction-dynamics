\chapter{DD Epistemology: What Can Be Known}

\section{The Problem of Knowledge}

How do we know anything?

Traditional answers:
\begin{itemize}
\item \textbf{Rationalism}: Reason provides knowledge (Descartes, Leibniz)
\item \textbf{Empiricism}: Experience provides knowledge (Locke, Hume)
\item \textbf{Kant}: Both, structured by mind's categories
\end{itemize}

DD answer: Knowledge = invariants of distinction structures.

\section{What Is Knowledge?}

\subsection{Traditional Definition}

Justified true belief (JTB).

Problems: Gettier cases, skeptical challenges.

\subsection{DD Definition}

Knowledge = stable distinction pattern that tracks other stable patterns.

\[
K(X) \iff \Delta_{\text{belief}}(X) \cong \Delta_{\text{reality}}(X)
\]

Isomorphism between internal and external distinction structures.

\section{Truth}

\subsection{Theories of Truth}

\begin{itemize}
\item Correspondence: Truth = matching reality
\item Coherence: Truth = fitting with other beliefs
\item Pragmatic: Truth = what works
\end{itemize}

\subsection{DD View}

Truth = structural correspondence of distinctions.

A belief is true when its distinction structure mirrors the distinction structure of what it's about.

This includes correspondence (matching reality), coherence (fitting other beliefs), and pragmatism (working) as special cases.

\section{Justification}

\subsection{The Regress Problem}

Every justification needs justification. Infinite regress?

\begin{itemize}
\item Foundationalism: Some beliefs need no justification
\item Coherentism: Beliefs justify each other circularly
\item Infinitism: Infinite chain is okay
\end{itemize}

\subsection{DD View}

$\Delta \neq \varnothing$ is self-justifying (using it to deny it confirms it).

From this foundation, justification proceeds by distinction preservation:

\[
A \text{ justifies } B \iff \Delta(A) \to \Delta(B) \text{ preserves structure}
\]

\section{A Priori and A Posteriori}

\subsection{Traditional Distinction}

A priori: Known independent of experience (math, logic)

A posteriori: Known through experience (science)

\subsection{DD View}

A priori = following from $\Delta \neq \varnothing$ alone.

A posteriori = depending on which $\Delta$ actually occur.

\begin{center}
\begin{tabular}{l|l}
\textbf{A Priori} & \textbf{A Posteriori} \\
\hline
Triadic structure & Which particles exist \\
Logical laws & Physical constants' values \\
Mathematical structures & Initial conditions \\
\end{tabular}
\end{center}

DD expands the a priori: More follows from $\Delta \neq \varnothing$ than traditionally thought.

\section{The Limits of Knowledge}

\subsection{Kant's Critique}

We cannot know things-in-themselves, only appearances structured by our categories.

\subsection{DD Reformulation}

We cannot know distinctions we cannot make.

Knowledge is limited by:
\begin{enumerate}
\item Our distinction-making capacity (cognitive limits)
\item Available distinctions (what exists to distinguish)
\item Time (distinctions take time to make)
\end{enumerate}

\subsection{Unknowable vs. Unknown}

\textbf{Unknown}: Distinctions we haven't made yet.

\textbf{Unknowable}: Distinctions that cannot be made (logical impossibilities, or beyond our capacity).

\section{The Asymptotic Nature of Knowledge}

\subsection{The Race (from Prologue)}

\begin{align}
\frac{dK}{dt} &> 0 \quad \text{(we learn)} \\
\frac{dR}{dt} &> 0 \quad \text{(reality complexifies)}
\end{align}

If $\frac{dR}{dt} \geq \frac{dK}{dt}$: asymptotic approach, never complete.

\subsection{Implications}

\begin{itemize}
\item No final theory (always more to distinguish)
\item Progress is real (we do learn)
\item Humility is warranted (we never arrive)
\end{itemize}

\section{Scientific Knowledge}

\subsection{What Makes Science Special?}

\begin{enumerate}
\item Systematic distinction-making (methodology)
\item Intersubjective verification (shared distinctions)
\item Prediction and control (distinction correspondence tested)
\end{enumerate}

\subsection{Scientific Revolutions}

Kuhn: Paradigm shifts are discontinuous.

DD: Paradigm shifts = restructuring of fundamental distinctions.

Newton $\to$ Einstein: Distinction between space and time collapsed into spacetime.

\section{Mathematical Knowledge}

\subsection{The Problem}

How do we know mathematical truths? They're not empirical.

\subsection{DD View}

Mathematics = study of distinction structures themselves.

$2 + 2 = 4$ because the distinction structure of two pairs equals the distinction structure of four units.

Mathematical knowledge = knowledge of what follows from $\Delta \neq \varnothing$.

\section{Self-Knowledge}

\subsection{The Puzzle}

Can I know myself? I'm the one doing the knowing.

\subsection{DD View}

Self-knowledge = $\Delta(\Delta)$.

Necessarily incomplete: The act of self-knowing creates new content to know.

\begin{center}
\fbox{\parbox{0.8\textwidth}{
Complete self-knowledge is impossible---not from contingent limits but from the structure of $\Delta(\Delta)$.

Self-knowing generates what must then be known.
}}
\end{center}

\section{Knowledge of Other Minds}

\subsection{The Problem}

How do I know others are conscious? I only experience my own mind.

\subsection{DD View}

Other minds = other $\Delta(\Delta)$ structures.

We infer them by:
\begin{enumerate}
\item Behavioral similarity (same distinction responses)
\item Structural similarity (same neural architecture)
\item Communication (distinction exchange)
\end{enumerate}

Not certain knowledge, but reasonable inference based on distinction correspondence.

\section{Skepticism}

\subsection{Radical Doubt}

Descartes: Maybe I'm deceived about everything.

\subsection{DD Response}

Even the deceiver must make distinctions.

Deception = substituting false distinctions for true ones.

The structure of distinction remains even if content is wrong.

\[
\text{``I am deceived''} \Rightarrow \text{``Distinctions exist (even if wrong)''}
\]

\section{Relativism}

\subsection{The Challenge}

Different cultures have different ``truths.'' Is truth relative?

\subsection{DD Response}

Distinctions are made from perspectives. But:
\begin{itemize}
\item Some distinctions are universal (triadic structure)
\item Correspondence can be tested (prediction)
\item Progress is measurable (more distinctions tracked)
\end{itemize}

Truth is not arbitrary even if perspective-dependent.

\section{Summary: DD Epistemology}

\begin{center}
\begin{tabular}{l|l}
\textbf{Concept} & \textbf{DD Interpretation} \\
\hline
Knowledge & Stable $\Delta$ tracking $\Delta$ \\
Truth & $\Delta$ correspondence \\
Justification & $\Delta$ preservation \\
A priori & From $\Delta \neq \varnothing$ alone \\
Limits & Cannot exceed $\Delta$-capacity \\
Science & Systematic $\Delta$-making \\
Math & Study of $\Delta$-structures \\
Self-knowledge & $\Delta(\Delta)$---incomplete \\
\end{tabular}
\end{center}

\begin{center}
\fbox{\parbox{0.8\textwidth}{
\textbf{DD Epistemology in one sentence:}

To know is to track distinctions with distinctions.

Knowledge is never complete but always real.
}}
\end{center}
