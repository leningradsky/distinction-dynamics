%==============================================================================
% Chapter: Necessity and Contingency
% Part I: Foundations
%==============================================================================

\chapter{Necessity and Contingency}\label{ch:necessity}

\epigraph{Nothing is accidental.}{---}

\section{Modal Definitions}

\begin{definition}[Modal Status]\label{def:modal}
A structure $S$ is:
\begin{itemize}
    \item \textbf{Necessary} if $\neg S \rightarrow \bot$ (its negation contradicts $\Delta$)
    \item \textbf{Impossible} if $S \rightarrow \bot$ (it contradicts $\Delta$)
    \item \textbf{Contingent} if both $S$ and $\neg S$ are consistent with $\Delta$
\end{itemize}
\end{definition}

\section{Elimination of Contingency}

\begin{theorem}[No Contingency]\label{thm:no-contingency}
No contingent structures exist.
\end{theorem}

\begin{proof}
Suppose $S$ is contingent: both $S$ and $\neg S$ are possible.
\begin{enumerate}
    \item In actuality, one of $\{S, \neg S\}$ obtains (call it $S$).
    \item $S$ obtaining = $S$ is distinguished as actual (vs.\ $\neg S$).
    \item This distinction is $\Delta$ (Lemma~\ref{lem:unity}).
    \item $\Self$ is self-determined (Axiom~\ref{ax:dd}).
    \item Self-determination means: $\Delta$'s structure fixes its outputs.
    \item Therefore, the distinction ``$S$ actual'' is fixed by $\Delta$'s structure.
    \item ``Fixed by $\Delta$'s structure'' = $\neg S$ contradicts $\Delta$.
    \item Therefore, $S$ is necessary (Definition~\ref{def:modal}).
    \item $S$ is not contingent. \qedhere
\end{enumerate}
\end{proof}

\begin{remark}
The key is that $\Self$ means there is no ``slack'' in $\Delta$. Every output is fixed. Contingency would require slack, but self-determination eliminates this.
\end{remark}

\begin{corollary}[Dichotomy]\label{cor:dichotomy}
Every structure is either necessary or impossible.
\end{corollary}

\begin{corollary}[Principle of Sufficient Reason]\label{cor:psr}
Everything that exists has a sufficient reason: it follows from $\Delta$. Everything that does not exist has a sufficient reason: it contradicts $\Delta$.
\end{corollary}

\section{The Leibnizian Question}

\begin{theorem}[Something Rather Than Nothing]\label{thm:something}
``Nothing'' is impossible. ``Something'' is necessary.
\end{theorem}

\begin{proof}
\begin{enumerate}
    \item ``Nothing'' = $\Delta = \emptyset$.
    \item But $\Self \neq \emptyset$ (Corollary~\ref{cor:exists}).
    \item Therefore, ``nothing'' contradicts $\Delta$.
    \item By Definition~\ref{def:modal}, ``nothing'' is impossible.
    \item By Corollary~\ref{cor:dichotomy}, ``something'' is necessary. \qedhere
\end{enumerate}
\end{proof}

\section{Uniqueness}

\begin{theorem}[Uniqueness]\label{thm:uniqueness}
For each type of structure, at most one realization exists.
\end{theorem}

\begin{proof}
\begin{enumerate}
    \item Suppose $S_1$ and $S_2$ are distinct realizations of type $T$.
    \item Both are actual (hypothesis) and thus necessary (Theorem~\ref{thm:no-contingency}).
    \item $S_1 \neq S_2$ requires a distinguishing criterion $C$.
    \item $C$ is either determined by $\Delta$ or external.
    \item External determination requires a legislator. Impossible (Theorem~\ref{thm:no-legislator}).
    \item Therefore, $C$ is determined by $\Delta$.
    \item $\Delta$ determines $S_1$ (with $C$) as necessary.
    \item Then $S_2$ (with $\neg C$) contradicts $\Delta$: impossible.
    \item Two distinct realizations cannot both be necessary.
    \item At most one exists. \qedhere
\end{enumerate}
\end{proof}

\begin{corollary}[Uniqueness of Reality]
There is exactly one universe, one set of laws, one set of constants.
\end{corollary}

\begin{corollary}[No Multiverse]
Multiple universes with different constants are impossible.
\end{corollary}

\begin{proof}
Different constants = different realizations of ``universe.'' By Theorem~\ref{thm:uniqueness}, at most one exists.
\end{proof}
