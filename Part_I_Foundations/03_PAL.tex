%==============================================================================
% Chapter: The Principle of the Absence of Legislator
% Part I: Foundations
%==============================================================================

\chapter{The Principle of the Absence of Legislator}\label{ch:pal}

\epigraph{No one is behind the curtain.}{---}

\section{The Legislator Concept}

\begin{definition}[Legislator]\label{def:legislator}
A \textbf{legislator} $L$ for a system $S$ is an agent such that:
\begin{enumerate}[label=(\roman*)]
    \item $L$ is external to $S$
    \item $L$ selects the structure of $S$ from alternatives $\{S_1, S_2, \ldots\}$
    \item $L$ is ontologically prior to $S$
\end{enumerate}
\end{definition}

\begin{remark}
``Ontologically prior'' means: $L$'s existence does not depend on $S$, but $S$'s structure depends on $L$'s choice. This covers atemporal ``choice'' as well as temporal creation.
\end{remark}

\section{Impossibility of Legislator}

\begin{theorem}[No Legislator — From Closure]\label{thm:no-legislator}
No legislator exists.
\end{theorem}

\begin{proof}
\begin{enumerate}
    \item Suppose $L$ is a legislator for $S$.
    \item By Definition~\ref{def:legislator}(i), $L$ is external to $S$.
    \item Take $S = \Delta$ (all of reality, since $\Delta$ contains all by Theorem~\ref{thm:closure}).
    \item Then $L$ is external to $\Delta$.
    \item By Theorem~\ref{thm:closure}, nothing exists outside $\Delta$.
    \item Contradiction.
    \item Therefore, no legislator exists for $\Delta$.
    \item Since every $S \subseteq \Delta$, no legislator exists for any $S$. \qedhere
\end{enumerate}
\end{proof}

\begin{theorem}[No Legislator — From Self-Reference]\label{thm:pal2}
No legislator exists (independent proof).
\end{theorem}

\begin{proof}
\begin{enumerate}
    \item Suppose $L$ selects structure $S$ from alternatives.
    \item Selection = distinguishing $S$ from alternatives $\{S_i\}$.
    \item Distinguishing is an act of $\Delta$ (Lemma~\ref{lem:unity}).
    \item Therefore, $L$'s act of selection \emph{is} an instance of $\Delta$.
    \item For $L$ to be ontologically prior to $S$, $L$ must be prior to $\Delta$ (since $S$ involves $\Delta$).
    \item But $L$'s act uses $\Delta$ (step 4).
    \item $L$ cannot be prior to what $L$ uses.
    \item Therefore, $L$ is not ontologically prior to $\Delta$.
    \item Definition~\ref{def:legislator}(iii) fails.
    \item $L$ is not a legislator. \qedhere
\end{enumerate}
\end{proof}

\begin{corollary}[Principle of the Absence of Legislator (PAL)]
No external agent determines the structure of reality. Reality is self-determining.
\end{corollary}

\section{Consequences of PAL}

\subsection{No Creator God}

\begin{corollary}\label{cor:no-god}
A God who chooses the structure of reality cannot exist.
\end{corollary}

\begin{proof}
Such a God would satisfy Definition~\ref{def:legislator}. By Theorem~\ref{thm:no-legislator}, no such entity exists.
\end{proof}

\begin{remark}
This does not preclude all concepts of divinity. One may identify $\Delta$ with the divine: self-existent, necessary, ground of all being, conscious (Chapter~\ref{ch:time-consc}). What is excluded is a God who ``decides'' from outside---a cosmic legislator.
\end{remark}

\subsection{No Ontological Randomness}

\begin{definition}[Ontological Randomness]\label{def:random}
An event $E$ is \textbf{ontologically random} if the outcome $O$ that occurs (rather than alternatives $\{O_i\}$) is not determined by any structure.
\end{definition}

\begin{theorem}[No Ontological Randomness]\label{thm:no-random}
No event is ontologically random.
\end{theorem}

\begin{proof}
\begin{enumerate}
    \item Suppose event $E$ has outcome $O$ that is ontologically random.
    \item $O$ occurring (rather than $O'$) is a distinction: $O$ vs.\ $\neg O$.
    \item By Lemma~\ref{lem:unity}, this distinction is $\Delta$.
    \item $\Delta = \Delta(\Delta)$ is a self-determined structure (Axiom~\ref{ax:dd}).
    \item A self-determined structure is not undetermined.
    \item Therefore, the distinction $O$ vs.\ $\neg O$ is determined by $\Delta$'s structure.
    \item Therefore, $O$ is not ontologically random. \qedhere
\end{enumerate}
\end{proof}

\begin{remark}
$\Self$ is a \emph{self-determining} fixed point. Self-determination excludes randomness by definition.
\end{remark}

\begin{corollary}[Quantum Mechanics]\label{cor:qm-epistemic}
Quantum randomness is epistemic, not ontological.
\end{corollary}

\subsection{No Arbitrary Initial Conditions}

\begin{corollary}\label{cor:no-ic}
The initial conditions of the universe are not arbitrary.
\end{corollary}

\begin{proof}
Arbitrary = selected without determining structure. Selection requires either a legislator (impossible) or determination by $\Delta$. Therefore, initial conditions are determined by $\Delta$.
\end{proof}

\section{Self-Sufficiency of Distinction}

\begin{theorem}[Self-Sufficiency]\label{thm:self-sufficient}
$\Delta$ requires no external ground.
\end{theorem}

\begin{proof}
Suppose $\Delta$ requires a ground $B$.
\begin{enumerate}
    \item $B$ either contains distinction or does not.
    \item \textbf{Case 1:} $B$ contains distinction $\Rightarrow$ $B$ presupposes $\Delta$.
    \item \textbf{Case 2:} $B$ does not contain distinction $\Rightarrow$ $B = \emptyset$.
    \item $\emptyset$ cannot generate $\Delta \neq \emptyset$.
    \item Both cases fail. No ground $B$ exists.
    \item $\Delta$ is self-sufficient. \qedhere
\end{enumerate}
\end{proof}

\begin{remark}
$\Delta$ is not unexplained. It is self-explanatory: its non-existence is self-contradictory.
\end{remark}
