\chapter*{Prologue: The Asymptotic Nature of Knowledge}
\addcontentsline{toc}{chapter}{Prologue}

\begin{flushright}
\textit{``To see everything is to see nothing.''}
\end{flushright}

\section*{The Cut}

Every proof is a cut. Every theory is an angle of view. Every understanding is a projection.

We do not see the whole. We see \emph{through} distinction. $\Delta$ is the lens through which reality becomes knowable.

Why must it be so? Because to see \emph{everything} is to see \emph{nothing}. Infinite information equals zero information. Distinction is the choice of what to see and what not to see.

$\Delta$ creates a boundary. Boundary creates form. Form is knowledge.

\section*{Consciousness as Integrator}

The present moment is a point. Contemplation is a point.

But consciousness sees the past through memory. Consciousness sees the future through anticipation. Consciousness sees the possible through imagination.

Consciousness is not a point---it is a \emph{volume}. An integral over time.

We see more than what exists now, because we ourselves are not merely ``now.'' A wave might think itself separate, but it is the movement of the ocean.

\section*{The Frame}

\begin{center}
\begin{tabular}{c|c}
\textbf{With frames ($\Delta$)} & \textbf{Without frames} \\
\hline
Order & Chaos \\
Distinguishability & Indifferentiation \\
Knowledge & Noise \\
Time & Eternity \\
Structure & Collapse
\end{tabular}
\end{center}

Remove one boundary, and neighboring boundaries lose their support. Remove those, and their neighbors collapse. An avalanche.

Everything begins to resonate with everything. Structure collapses into the undifferentiated.

This is not metaphor. It is literal:
\begin{itemize}
\item Psychosis = loss of the boundary between self and not-self
\item Death = loss of the boundary between organism and environment  
\item Heat death = loss of all thermodynamic gradients
\end{itemize}

\section*{Why Not Total Collapse?}

The axiom of this work:
\[
\Delta \neq \varnothing
\]

Distinction \emph{exists}. It cannot not exist. Because ``no distinction'' is itself a distinction.

Cascading resonance cannot be complete. At least one $\Delta$ will always remain. From it---new structure.

\section*{The Evolution of the Universe}

\begin{center}
\begin{tabular}{r@{ = }l}
Big Bang & minimum distinctions (near-homogeneity) \\
Expansion & growth of distinctions \\
Structures & local maxima of distinguishability \\
Life & autocatalytic $\Delta$ \\
Consciousness & $\Delta(\Delta)$
\end{tabular}
\end{center}

The universe \emph{accumulates} distinctions. Entropy grows, but so does structure. These are not contradictory---they are two aspects of the same process.

\section*{Our Cut}

We are a local point in this process. Our knowledge is a cut through accumulated distinctions. It seems absolute from the inside. From outside, it is part of the flow.

Like a wave thinking itself separate, while being the motion of the ocean.

\section*{The Asymptote of Knowledge}

Complete knowledge would mean seeing all $\Delta$. But we ourselves are the result of $\Delta$. To see all distinctions would be to see ourselves from outside---a contradiction.

Knowledge asymptotically approaches. It never arrives. But the \emph{process} of approaching \emph{is} consciousness.

Let $K(t)$ be knowledge at time $t$, and $R(t)$ be reality at time $t$:
\begin{align}
\frac{dK}{dt} &> 0 \quad \text{(we learn)} \\
\frac{dR}{dt} &> 0 \quad \text{(reality complexifies)}
\end{align}

If $\frac{dR}{dt} \geq \frac{dK}{dt}$, then $K(t) < R(t)$ always. An asymptote.

But: $\frac{K(t)}{R(t)} \to c > 0$. We do not fall infinitely behind. We maintain \emph{proportion}.

Consciousness does not catch up to reality, but maintains connection with it.

\section*{The Race}

To solve a problem requires time $T$. During time $T$, the problem has grown by $\Delta T$.

If $\Delta T >$ what was solved in $T$: never catch up. Asymptote.

This is not failure. This is the \emph{structure} of knowledge itself.

\begin{center}
\textbf{P problems}: Generation $<$ Elimination. You catch up. You solve.

\textbf{NP problems}: Generation $\geq$ Elimination. A race without finish.
\end{center}

The triad is the minimal autocatalytic structure:

\begin{center}
\begin{tikzpicture}[scale=1.5]
\node (a) at (0,0) {$a$};
\node (b) at (2,0) {$b$};
\node (c) at (1,1.732) {$c$};
\draw[<->] (a) -- (b);
\draw[<->] (b) -- (c);
\draw[<->] (c) -- (a);
\end{tikzpicture}
\end{center}

Distinguish $a$ from $b$---must position relative to $c$. Distinguish $b$ from $c$---must recheck $a$. A closed cycle of generation.

Each distinction within the triad triggers others. This is why triadic structure marks the boundary of tractability.

\section*{Invariants}

The cut does not equal the whole. But the cut can contain \emph{invariants} of the whole.

$\Delta \neq \varnothing$ is an invariant. The triad is an invariant. Closure is an invariant.

We do not see everything. We see what \emph{does not change} while everything changes.

\section*{What This Book Contains}

This book is itself a cut. A particular angle through the space of possible theories.

We will derive:
\begin{itemize}
\item Why there are three spatial dimensions (\textbf{Part III})
\item Why the gauge group is $SU(3) \times SU(2) \times U(1)$ (\textbf{Part III})
\item Why consciousness has the structure it has (\textbf{Part IV})
\item Why P $\neq$ NP (probably) (\textbf{Part II})
\end{itemize}

All from the single axiom: $\Delta \neq \varnothing$.

Not because we have found the final truth. But because we have found an \emph{invariant}---something that remains while everything else changes.

The cut is not the whole. But if the cut is made correctly, it reveals the structure of the whole.

\vspace{1cm}
\begin{flushright}
\textit{``We cannot know everything.\\
We can know the form of everything.\\
Form is distinction.''}
\end{flushright}
