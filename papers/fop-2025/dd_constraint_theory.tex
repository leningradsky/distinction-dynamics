\documentclass[11pt,a4paper]{article}

% ===== PACKAGES =====
\usepackage[utf8]{inputenc}
\usepackage[T1]{fontenc}
\usepackage{amsmath,amssymb,amsthm}
\usepackage{mathrsfs}
\usepackage{booktabs}
\usepackage[colorlinks=true,linkcolor=blue,citecolor=blue,urlcolor=blue]{hyperref}
\usepackage{geometry}
\geometry{margin=1in}

% ===== THEOREM ENVIRONMENTS =====
\newtheorem{theorem}{Theorem}[section]
\newtheorem{proposition}[theorem]{Proposition}
\newtheorem{lemma}[theorem]{Lemma}
\newtheorem{corollary}[theorem]{Corollary}
\newtheorem{definition}[theorem]{Definition}
\newtheorem{example}[theorem]{Example}
\newtheorem{assumption}[theorem]{Assumption}
\newtheorem{remark}[theorem]{Remark}

% ===== TITLE =====
\title{\textbf{Distinction Dynamics as a Constraint Theory}}
\author{[Author]}
\date{\today}

\begin{document}

\maketitle

\begin{abstract}
We present Distinction Dynamics (DD) as a \emph{constraint theory}---a framework 
that delimits what structures can exist rather than predicting specific dynamics.
DD is positioned alongside established constraint theorems such as Coleman--Mandula,
CPT, and spin--statistics: results that restrict the space of consistent theories
without specifying a unique model. We provide a formal characterisation of
``distinction acts'' invariant under choice of mathematical carrier, demonstrate
that the uniqueness of Standard Model structure is robust under different cost
functionals, and show how CP violation connects to the arrow of time within the
DD framework. A counterexample (four-generation model) illustrates how DD excludes
alternatives constructively. The paper clarifies the methodological status of DD
and its relation to existing foundational results in physics.
\end{abstract}

\tableofcontents
\newpage

% ===================================================================
% SECTION 1: INTRODUCTION
% ===================================================================
\section{Introduction}

Physics contains two types of theoretical structures:
\begin{enumerate}
\item \textbf{Dynamical theories}: specify equations of motion, predict outcomes
      (e.g.\ QED, General Relativity, Standard Model)
\item \textbf{Constraint theories}: delimit what is possible without specifying
      dynamics (e.g.\ Coleman--Mandula, CPT theorem, spin--statistics)
\end{enumerate}

Constraint theories answer questions of the form ``why not X?'' rather than
``what happens next?'' They are not competitors to dynamical theories but
\emph{meta-structures} that restrict the space of consistent theories.

Distinction Dynamics (DD) belongs to the second category. Its core claim is:
\begin{quote}
\emph{Certain structural features of physics---including quantum theory, gauge
structure, and the number of fermion generations---follow from conditions on
what can exist as a coherent system of distinctions.}
\end{quote}

This paper clarifies the methodological status of DD by:
\begin{itemize}
\item Providing a formal, carrier-independent characterisation of distinction acts
\item Demonstrating robustness of uniqueness results under different cost measures
\item Connecting CP violation to the arrow of time within DD
\item Giving explicit counterexamples that DD excludes
\item Positioning DD relative to established constraint theorems
\end{itemize}

\subsection*{Contributions}

DD does not claim to derive new physics. Its contributions are:

\begin{enumerate}
\item \textbf{Logical unification}: DD connects operational axioms of QM 
      (Hardy, CDP reconstructions), algebraic constraints (Hurwitz), and 
      gauge-algebra correspondences (Furey, Gresnigt) into a single chain 
      with explicit logical dependencies.
      
\item \textbf{Saturation theorem}: We prove that the Standard Model structure 
      \emph{saturates} the space of realisable theories---extensions are either 
      non-realisable or merely effective (Theorem~\ref{thm:saturation}).
      
\item \textbf{Fundamental/effective distinction}: We provide a criterion for 
      separating UV-complete structure (constrained by DD) from emergent 
      effective symmetries (not constrained).
\end{enumerate}

These are contributions to the \emph{organisation} of known results, not 
claims of new empirical predictions.


% ===================================================================
% SECTION 2: FORMAL STRUCTURE
% ===================================================================
\section{Formal Characterisation of Distinction Acts}

\subsection{The Problem of Mathematical Carrier}

DD operates with the concept of ``distinction act''---a primitive notion that
admits multiple mathematical formalisations:
\begin{itemize}
\item Processes in a monoidal category
\item Effects in generalised probabilistic theories (GPT)
\item Elements of a normed algebra
\item Morphisms in a dagger category
\end{itemize}

The question arises: do DD's conclusions depend on the choice of formalism?

\subsection{Carrier-Independent Axioms}

We show that DD's core constraints can be stated in a form invariant under
carrier choice.

\begin{definition}[Abstract Distinction System]
An \emph{abstract distinction system} is a triple $(\mathcal{D}, \circ, \|\cdot\|)$
where:
\begin{enumerate}
\item $\mathcal{D}$ is a set of distinction acts
\item $\circ: \mathcal{D} \times \mathcal{D} \to \mathcal{D}$ is a composition
      (associative, with identity)
\item $\|\cdot\|: \mathcal{D} \to \mathbb{R}_{\geq 0}$ is a norm satisfying:
      \begin{enumerate}
      \item $\|a\| = 0 \Leftrightarrow a = 0$ (non-degeneracy)
      \item $\|a \circ b\| = \|a\| \|b\|$ (multiplicativity)
      \end{enumerate}
\end{enumerate}
\end{definition}

\begin{definition}[Realisability]
A distinction system is \emph{realisable} if every non-zero element is invertible:
\[
a \neq 0 \Rightarrow \exists a^{-1}: a \circ a^{-1} = a^{-1} \circ a = e.
\]
\end{definition}

\begin{theorem}[Carrier Independence]
Let $(\mathcal{D}, \circ, \|\cdot\|)$ be a realisable distinction system over
$\mathbb{R}$. Then $\mathcal{D}$ is isomorphic to one of $\mathbb{R}$, $\mathbb{C}$,
$\mathbb{H}$, or $\mathbb{O}$.
\end{theorem}

\begin{proof}
A realisable distinction system over $\mathbb{R}$ is precisely a normed division
algebra. By Hurwitz's theorem (1898), such algebras exist only in dimensions
$1, 2, 4, 8$, corresponding to $\mathbb{R}$, $\mathbb{C}$, $\mathbb{H}$, $\mathbb{O}$.
The specific choice of carrier (category, GPT, algebra) is irrelevant: any
structure satisfying the axioms must reduce to one of these four.
\end{proof}

\begin{corollary}
DD's structural constraints are independent of the mathematical language used
to express them.
\end{corollary}


\subsection{Saturation of Realisability}

The following theorem captures why SM is not merely ``minimal'' but 
\emph{saturated}---there is no room to add structure without violating 
realisability.

\begin{theorem}[Realisability Saturation]
\label{thm:saturation}
Let $T$ be a realisable structure with unique closure under the realisability
constraints. Then for any strict extension $T' \supsetneq T$:
\begin{enumerate}
\item Either $T'$ violates realisability (contains non-invertible elements), or
\item $T'$ is dynamically effective but not fundamentally distinct from $T$
      (the additional structure is emergent, not primitive).
\end{enumerate}
\end{theorem}

\begin{proof}[Argument]
Realisability constrains the carrier algebra to $\mathbb{R}, \mathbb{C}, 
\mathbb{H}, \mathbb{O}$ (Hurwitz). Given the physical requirement of complex
phases for interference, this selects $\mathbb{C} \otimes \mathbb{O}$.

The gauge structure $SU(3) \times SU(2) \times U(1)$ exhausts the automorphism
content of this algebra. The three generations exhaust the octonionic subalgebras
of $\mathbb{S}$ (sedenions serve as the closure, not the carrier).

Any extension must either:
\begin{itemize}
\item Add gauge factors $\Rightarrow$ requires larger algebra $\Rightarrow$ 
      sedenions or beyond $\Rightarrow$ zero divisors $\Rightarrow$ non-realisable
\item Add generations $\Rightarrow$ requires fourth octonionic subalgebra 
      $\Rightarrow$ does not exist $\Rightarrow$ non-realisable
\item Add fermion types (vector-like) $\Rightarrow$ violates chirality required
      by complex structure $\Rightarrow$ non-realisable at fundamental level
\end{itemize}

Extensions that appear consistent (e.g., effective $U(1)_{B-L}$) are emergent
symmetries, not additional primitive structure.
\end{proof}

\begin{remark}[Not Occam's Razor]
This is not a simplicity argument. We do not claim SM is ``simplest'' among
possible theories. We claim SM \emph{saturates} the space of realisable
structures---it uses all available algebraic room and leaves none.
\end{remark}

\subsection{Relation to Existing Formalisms}

\begin{center}
\begin{tabular}{lcc}
\toprule
Formalism & DD translation & Realisability \\
\midrule
Monoidal category & morphisms & invertible morphisms \\
GPT & effects & pure effects \\
C*-algebra & elements & unitary elements \\
Dagger category & morphisms & unitary morphisms \\
\bottomrule
\end{tabular}
\end{center}

In each case, the realisability condition selects the same substructure, and
Hurwitz's bound applies uniformly.


% ===================================================================
% SECTION 3: ROBUSTNESS OF COST FUNCTIONAL
% ===================================================================
\section{Robustness of the Cost Functional}

\subsection{The Question}

In the companion paper, we introduced a distinction cost functional:
\[
\mathcal{S}(T) = \alpha\,\mathrm{rank}(G) + \beta\,\dim(G) + \gamma\,N_{\text{chiral}}
+ \delta\,N_{\text{scalar}} + \epsilon\,N_{\text{free}}.
\]

A natural concern: does the minimality of the Standard Model depend on the
specific choice of coefficients $\alpha, \beta, \gamma, \delta, \epsilon$?

\subsection{Monotonicity Theorem}

\begin{definition}[Admissible Cost Functional]
A cost functional $\mathcal{S}: \mathfrak{T} \to \mathbb{R}_{\geq 0}$ is
\emph{admissible} if:
\begin{enumerate}
\item $\mathcal{S}(T) \geq 0$ for all $T$
\item $T' \supset T$ (strict extension) implies $\mathcal{S}(T') > \mathcal{S}(T)$
\item $\mathcal{S}$ depends only on structural data (group, representations, scalars)
\end{enumerate}
\end{definition}

\begin{proposition}[Robustness of Minimum]
Let $\mathcal{S}_1, \mathcal{S}_2$ be two admissible cost functionals on the class
$\mathfrak{T}$ of realisable, anomaly-free theories with minimal scalar content.
If a theory $T^*$ is the unique minimum of $\mathcal{S}_1$, then $T^*$ is also
the unique minimum of $\mathcal{S}_2$.
\end{proposition}

\begin{proof}
By the uniqueness theorem, the class $\mathfrak{T}$ contains exactly one element
satisfying all constraints: the Standard Model. Since any extension strictly
increases any admissible $\mathcal{S}$, the Standard Model is the unique minimum
regardless of which admissible functional is used.
\end{proof}

\begin{corollary}
The specific coefficients in $\mathcal{S}(T)$ are irrelevant. Any admissible
measure yields the same conclusion: SM is the unique minimum.
\end{corollary}

This is analogous to computational complexity: different reasonable complexity
measures may disagree on constants but agree on which problems are tractable.


% ===================================================================
% SECTION 4: CP VIOLATION AND THE ARROW OF TIME
% ===================================================================
\section{CP Violation and the Arrow of Time}

\subsection{The Puzzle}

In the main paper, CP violation serves as the lower bound for $N_{\text{gen}} \geq 3$.
But this raises a deeper question: why is CP violation possible at all?

Within DD, distinctions are \emph{a priori} symmetric: if $A$ can be distinguished
from $B$, then $B$ can be distinguished from $A$. Yet the physical world exhibits
temporal asymmetry.

\subsection{Resolution: Realisation Requires Orientation}

\begin{proposition}[Orientation from Realisation]
Let $\mathcal{D}$ be an abstract distinction system. The act of \emph{realising}
a distinction (making it physical) requires selecting an orientation:
which state is ``before'' and which is ``after'' the distinction act.
\end{proposition}

\textit{Argument.} A distinction act $a \in \mathcal{D}$ is abstract until embedded
in a causal structure. Embedding requires:
\begin{enumerate}
\item A notion of ``input'' and ``output'' states
\item A temporal ordering: input precedes output
\end{enumerate}

This ordering is not intrinsic to $\mathcal{D}$ but emerges from realisation.
The choice of orientation breaks the symmetry between $a$ and $a^{-1}$.

\begin{corollary}
CP violation is the \emph{minimal trace} of temporal orientation in a
realised distinction system.
\end{corollary}

\textit{Explanation.} In a system with:
\begin{itemize}
\item Complex structure (required for quantum interference)
\item Multiple generations (required for mixing)
\end{itemize}
the CKM matrix acquires a complex phase. This phase is physically detectable
precisely because the system has a temporal orientation. With only two generations,
no phase survives---no trace of orientation remains in flavour physics.

\subsection{Connection to Thermodynamics}

This connects DD to the thermodynamic arrow:
\begin{itemize}
\item Distinction = information about difference
\item Realisation = physical instantiation of information
\item Temporal orientation = direction of entropy increase
\end{itemize}

CP violation is thus not an ``accident'' but the \emph{necessary signature} of
a realised distinction system with temporal structure.


% ===================================================================
% SECTION 5: MASS AS DEPTH OF DISTINCTION
% ===================================================================
\section{Mass as Depth of Distinction (Structural Account)}
\label{sec:mass-depth}

\subsection{The Hierarchy as Structural Fact}

The Standard Model exhibits a striking mass hierarchy:
\[
m_e : m_\mu : m_\tau \approx 1 : 200 : 3500
\]
\[
m_u : m_c : m_t \approx 1 : 500 : 75000
\]

We do \emph{not} aim to predict numerical masses. Instead we ask for a
\emph{structural} statement: why a robust \emph{ordering} (light $\prec$ heavy)
is natural once the theory is viewed as a hierarchy of realised distinctions.

\subsection{Filtration by Depth}

\begin{definition}[Distinction Depth and Filtration]
Let $\mathcal{D}$ be the set of realisable distinction acts.
A \emph{depth function} is a map $\mathrm{depth}:\mathcal{D}\to \mathbb{N}$
such that the induced subsets
\[
\mathcal{D}_{\le k}:=\{a\in\mathcal{D}\,:\,\mathrm{depth}(a)\le k\}
\]
form an increasing filtration $\mathcal{D}_{\le 0}\subset \mathcal{D}_{\le 1}
\subset \cdots$, interpreted as ``distinctions requiring at most $k$ nested
levels of stabilisation''.
\end{definition}

For fermion generations:
\begin{itemize}
\item $\mathcal{D}_{\le 0}$: gauge distinctions (colour, weak isospin, hypercharge)
\item $\mathcal{D}_{\le 1}$: first generation (electron, up, down)
\item $\mathcal{D}_{\le 2}$: second generation (muon, charm, strange)
\item $\mathcal{D}_{\le 3}$: third generation (tau, top, bottom)
\end{itemize}


\subsection{RG Interpretation: Depth as Operator Complexity}

In effective field theory, physical distinctions correspond to operators in
the action. Under Wilsonian renormalisation, operators are classified by
scaling dimension $\Delta$ into relevant/marginal/irrelevant, controlling
their infrared impact \cite{WilsonKogut1974,Polchinski1984}.

\begin{definition}[Mass-Generating Operator Classes]
For each particle species $x$, we identify a dominant mass-generating operator:
\begin{center}
\begin{tabular}{lll}
\toprule
Particle type & Operator class & Schematic form \\
\midrule
Charged leptons & Yukawa & $y_\ell \bar{L} H e_R$ \\
Up-type quarks & Yukawa & $y_u \bar{Q} \tilde{H} u_R$ \\
Down-type quarks & Yukawa & $y_d \bar{Q} H d_R$ \\
Neutrinos (Dirac) & Yukawa & $y_\nu \bar{L} \tilde{H} \nu_R$ \\
Neutrinos (Majorana) & Weinberg & $\frac{1}{\Lambda}(LH)(LH)$ \\
\bottomrule
\end{tabular}
\end{center}
\end{definition}

\begin{assumption}[Depth--Dimension Monotonicity]
\label{ass:depth-dim}
There exists an assignment of particle species $x$ to dominant mass-generating
operators $\mathcal{O}_x$ such that
\[
\mathrm{depth}(x) < \mathrm{depth}(y)\quad \Longrightarrow\quad
\Delta_{\text{eff}}(\mathcal{O}_x) \le \Delta_{\text{eff}}(\mathcal{O}_y),
\]
where $\Delta_{\text{eff}} = \Delta_{\text{classical}} + \gamma$ includes
anomalous dimension $\gamma$ from quantum corrections.
\end{assumption}

\textit{Physical interpretation.} Deeper distinctions require more contextual
support---more ``conditions to be satisfied'' for the distinction to be stable.
In RG language, this corresponds to operators with more fields or derivatives,
hence higher classical dimension, or operators that receive large anomalous
corrections due to strong coupling to the stabilising sector.


\subsection{Monotonicity Result}

\begin{proposition}[Conditional Monotonicity of Mass--Depth]
\label{thm:mass-depth}
Within any framework satisfying:
\begin{enumerate}
\item Particle masses arise from couplings of renormalised operators under RG flow
\item The depth--dimension monotonicity (Assumption~\ref{ass:depth-dim}) holds
\end{enumerate}
the mass ordering is monotone with depth: for $x\in \mathcal{D}_{\le k}$ and
$y\in \mathcal{D}_{\le k+1}\setminus \mathcal{D}_{\le k}$,
\[
m(y) \gtrsim m(x),
\]
with strict inequality generically unless protected by symmetry.
\end{proposition}

\begin{proof}[Proof sketch]
In Wilsonian EFT, couplings run according to their RG eigenvalues determined
by operator dimensions \cite{WilsonKogut1974,Polchinski1984}. A mass parameter
is the coefficient of the leading mass-generating operator after symmetry breaking.

If deeper distinctions correspond to operators with larger effective dimension,
then maintaining them as stable low-energy degrees of freedom requires larger
matching coefficients at the UV scale. Under RG flow to the IR, these translate
to larger physical masses.

More precisely: let $\mathcal{O}_x$ have dimension $\Delta_x$ and $\mathcal{O}_y$
have dimension $\Delta_y > \Delta_x$. The ratio of induced masses scales as
\[
\frac{m_y}{m_x} \sim \left(\frac{\Lambda_{\text{UV}}}{\Lambda_{\text{IR}}}\right)^{\Delta_y - \Delta_x} \gg 1
\]
for $\Delta_y > \Delta_x$ and hierarchically separated scales.
\end{proof}


\subsection{Symmetry Protection and Exceptions}

The monotonicity is \emph{generic but not absolute}. Known mechanisms that
can modify or violate the naive ordering:

\begin{center}
\begin{tabular}{lll}
\toprule
Mechanism & Effect & Example \\
\midrule
Chiral symmetry & Suppresses mass & Light quarks ($u$, $d$, $s$) \\
Approximate flavour symmetry & Near-degeneracy & $m_u \approx m_d$ \\
See-saw mechanism & Inverts hierarchy & Neutrino masses \\
Technical naturalness & Protects lightness & Electron vs.\ QCD scale \\
\bottomrule
\end{tabular}
\end{center}

These exceptions do not violate the theorem; they correspond to additional
symmetry structure that modifies the effective dimension or protects certain
operators from receiving large corrections.

\begin{remark}[Neutrinos]
Neutrino masses appear to violate the generation ordering ($m_{\nu_3}$ largest
but still $\ll m_e$). This is explained by the see-saw mechanism: the Weinberg
operator has dimension 5, not 4, introducing an additional suppression factor
$v^2/\Lambda$ where $\Lambda \gg v$ is the heavy right-handed neutrino scale.
The \emph{intra-neutrino} ordering still respects depth monotonicity.
\end{remark}

\subsection{What This Does and Does Not Claim}

\begin{center}
\begin{tabular}{ll}
\toprule
DD claims & DD does not claim \\
\midrule
Mass hierarchy is structurally natural & Numerical mass values \\
Ordering is RG-robust & Precise ratios $m_\tau/m_\mu$ \\
Deeper $\Rightarrow$ heavier (generically) & Yukawa coupling values \\
Exceptions require symmetry protection & Origin of flavour symmetries \\
\bottomrule
\end{tabular}
\end{center}


The mass hierarchy becomes readable as an emergent grading: deeper realised
distinctions require more stabilisation under renormalisation. This answers
the objection ``you do not explain the hierarchy'' honestly: we explain
\emph{why hierarchy is structurally natural} and why its ordering is
RG-robust, while acknowledging that numerical values require dynamics.

% ===================================================================
% SECTION 6: COUNTEREXAMPLES
% ===================================================================
\section{Counterexamples: What DD Excludes}

A constraint theory gains credibility by exhibiting what it \emph{excludes}.
We construct explicit models that DD forbids.

\subsection{Four Fermion Generations}

Consider a hypothetical Standard Model with four fermion generations:
\begin{itemize}
\item Gauge group: $SU(3) \times SU(2) \times U(1)$ (unchanged)
\item Fermions: four copies of $(Q_L, u_R, d_R, L_L, e_R)$
\item Anomaly cancellation: satisfied (each generation is anomaly-free)
\end{itemize}

This model is:
\begin{itemize}
\item Perturbatively consistent
\item Anomaly-free
\item Not experimentally excluded (except by precision data)
\end{itemize}

\subsection{DD Exclusion}

\begin{theorem}[Four Generations Excluded]
A Standard Model with four fermion generations violates realisability.
\end{theorem}

\begin{proof}
By the sedenion structure theorem (Proposition 6.4 of the main paper), the
algebra $\mathbb{S}$ contains exactly three maximal division-preserving
subalgebras. A fourth generation would require a fourth such subalgebra.

Explicitly: the automorphism group $\mathrm{Aut}(\mathbb{S}) = G_2 \times S_3$
acts on octonionic subalgebras. The $S_3$ factor permutes exactly three
subalgebras; there is no $S_4$ action. A fourth generation has no algebraic
home in the realisability structure.
\end{proof}

\begin{corollary}
The four-generation model is anomaly-free but \emph{non-realisable}.
Anomaly cancellation is necessary but not sufficient for physical existence.
\end{corollary}


\subsection{Significance}

This counterexample demonstrates that DD is not merely a restatement of known
constraints. It provides an \emph{independent} exclusion mechanism:

\begin{center}
\begin{tabular}{lcc}
\toprule
Constraint & Four generations & Status \\
\midrule
Anomaly cancellation & Satisfied & --- \\
Perturbative consistency & Satisfied & --- \\
Electroweak precision & Violated & Empirical \\
\textbf{DD realisability} & \textbf{Violated} & \textbf{Structural} \\
\bottomrule
\end{tabular}
\end{center}

DD excludes four generations \emph{a priori}, without reference to experiment.

\subsection{Second Counterexample: SM + Extra $U(1)'$}

Consider extending the Standard Model by an additional abelian factor:
\[
SU(3) \times SU(2) \times U(1)_Y \times U(1)'
\]
with fermions charged under $U(1)'$ in a generation-dependent pattern
(e.g., $B-L$ or family-specific charges).

\textbf{Status by standard criteria:}
\begin{itemize}
\item Anomaly cancellation: can be satisfied with appropriate charge assignments
\item Perturbative consistency: satisfied
\item Phenomenologically viable: constrained but not excluded
\end{itemize}


\begin{theorem}[Extra $U(1)$ Excluded by Realisability]
Any extension $SU(3) \times SU(2) \times U(1) \times U(1)'$ with fermions
charged under $U(1)'$ violates the minimality implied by realisability.
\end{theorem}

\begin{proof}[Argument]
The gauge structure $SU(3) \times SU(2) \times U(1)$ emerges uniquely from
$\mathbb{C} \otimes \mathbb{O}$ (Furey 2018). This algebraic derivation
produces \emph{exactly one} $U(1)$ factor---the hypercharge.

An additional $U(1)'$ would require either:
\begin{enumerate}
\item A second independent phase rotation in the algebra, or
\item An extension beyond $\mathbb{C} \otimes \mathbb{O}$
\end{enumerate}

Option (1) is impossible: the single $U(1)$ saturates the abelian content
of the octonionic automorphism structure.

Option (2) requires moving to sedenions or higher Cayley--Dickson algebras,
which contain zero divisors and violate realisability (non-invertibility).

Therefore, no realisable extension admits an extra $U(1)$ factor.
\end{proof}

\begin{remark}
This does \emph{not} exclude $U(1)_{B-L}$ as an \emph{emergent} symmetry
at low energies. DD constrains the \emph{fundamental} gauge structure,
not effective symmetries arising from dynamics.
\end{remark}


\subsection{Summary of Exclusions}

\begin{center}
\begin{tabular}{lccc}
\toprule
Model & Anomaly-free & Perturbative & DD-realisable \\
\midrule
SM (3 generations) & \checkmark & \checkmark & \checkmark \\
SM (4 generations) & \checkmark & \checkmark & $\times$ \\
SM + $U(1)'$ & \checkmark$^*$ & \checkmark & $\times$ \\
SM + vector-like fermions & \checkmark & \checkmark & $\times$$^\dagger$ \\
\bottomrule
\end{tabular}
\end{center}

{\small $^*$With appropriate charge assignment. $^\dagger$Violates chiral
structure required by realisability.}

DD is not fitted to SM. It \emph{derives} SM as the unique realisable structure
and \emph{excludes} natural-looking extensions.


% ===================================================================
% SECTION 7: POSITION AMONG CONSTRAINT THEOREMS
% ===================================================================
\section{DD Among Constraint Theorems}

\subsection{Classification of Constraint Theorems}

\begin{center}
\begin{tabular}{lll}
\toprule
Theorem & What it constrains & Key assumption \\
\midrule
Coleman--Mandula & Symmetry mixing & Lorentz + S-matrix \\
CPT & Discrete symmetries & QFT + Lorentz \\
Spin--statistics & Spin $\leftrightarrow$ statistics & QFT + locality \\
Weinberg--Witten & Massless higher spin & Lorentz + conserved currents \\
\textbf{DD} & \textbf{Gauge group, generations} & \textbf{Realisability} \\
\bottomrule
\end{tabular}
\end{center}

\subsection{Common Structure}

All constraint theorems share a logical form:
\[
\text{(General assumptions)} \Rightarrow \text{(Restricted class of theories)}
\]

DD fits this pattern:
\[
\text{(Realisability)} \Rightarrow \text{(SM-like structure)}
\]

The assumptions (unitarity, invertibility, norm conservation) are no stronger
than those of other constraint theorems.


\subsection{What DD Does Not Do}

\begin{itemize}
\item DD does not predict coupling constants
\item DD does not derive the Lagrangian
\item DD does not explain numerical mass values
\item DD does not replace dynamical theories
\end{itemize}

DD answers: ``Why this structure and not another?''

It does not answer: ``What happens in a specific experiment?''

This is the proper scope of a constraint theory.

\subsection{Fundamental vs.\ Effective Structure}

A crucial distinction for interpreting DD's claims:

\begin{definition}[Fundamental vs.\ Effective]
\begin{itemize}
\item A structure is \emph{fundamental} if it appears in the UV-complete 
      description of the theory
\item A structure is \emph{effective} if it emerges at some energy scale
      but is not required by the fundamental formulation
\end{itemize}
\end{definition}

\textbf{DD constrains fundamental structure only.}

\begin{center}
\begin{tabular}{lcc}
\toprule
Structure & Fundamental & Effective \\
\midrule
$SU(3) \times SU(2) \times U(1)$ & \checkmark & --- \\
Hypercharge quantisation & \checkmark & --- \\
Three chiral generations & \checkmark & --- \\
$U(1)_{B-L}$ & $\times$ & \checkmark \\
Approximate flavour symmetries & $\times$ & \checkmark \\
Custodial $SU(2)$ & $\times$ & \checkmark \\
\bottomrule
\end{tabular}
\end{center}


\begin{remark}[Why This Matters]
Effective symmetries may:
\begin{itemize}
\item Emerge from dynamics without being realisability-constrained
\item Be broken at higher energies without contradiction
\item Appear ``additional'' to SM without violating DD
\end{itemize}
DD does not forbid $U(1)_{B-L}$ as an effective low-energy symmetry.
It forbids $U(1)_{B-L}$ as a \emph{fundamental gauge factor} in the 
UV-complete theory.
\end{remark}

This distinction preempts the objection: ``But we observe approximate 
symmetries beyond SM gauge structure!'' Such symmetries are dynamical 
accidents, not structural necessities.

% ===================================================================
% SECTION 8: CONCLUSION
% ===================================================================
\section{Conclusion}

Distinction Dynamics is a constraint theory that delimits the space of
consistent physical structures. Its core results:

\begin{enumerate}
\item \textbf{Carrier independence}: DD's conclusions hold regardless of
      mathematical formalism (categories, GPT, algebras)
\item \textbf{Robust uniqueness}: The Standard Model is the unique minimum
      under any admissible cost functional
\item \textbf{Arrow of time}: CP violation is the minimal trace of temporal
      orientation in a realised distinction system
\item \textbf{Mass hierarchy}: Monotonicity theorem explains why deeper 
      generations are heavier; RG provides the mechanism
\item \textbf{Constructive exclusion}: Four generations, extra $U(1)$ factors,
      and vector-like fermions are ruled out structurally
\end{enumerate}

DD does not compete with dynamical theories. It establishes the boundaries
within which dynamics must operate---just as Coleman--Mandula establishes
boundaries for symmetry, and spin--statistics for particle identity.

The methodological status is clear: DD is not a ``theory of everything'' but
a \emph{theorem about structure}---showing that much of what appears contingent
in the Standard Model is in fact necessary.


% ===================================================================
% BIBLIOGRAPHY
% ===================================================================
\begin{thebibliography}{99}

\bibitem{ColemanMandula}
S.~Coleman and J.~Mandula, 
``All Possible Symmetries of the S Matrix,'' 
Phys.\ Rev.\ \textbf{159}, 1251 (1967).

\bibitem{Hurwitz1898}
A.~Hurwitz,
``\"Uber die Composition der quadratischen Formen von beliebig vielen Variablen,''
Nachr.\ Ges.\ Wiss.\ G\"ottingen (1898), 309--316.

\bibitem{Schafer1954}
R.~D.~Schafer, 
``On the algebras formed by the Cayley--Dickson process,'' 
Amer.\ J.\ Math.\ \textbf{76}, 435 (1954).

\bibitem{GengMarshak}
C.~Q.~Geng and R.~E.~Marshak, 
``Uniqueness of Quark and Lepton Representations in the Standard Model,'' 
Phys.\ Rev.\ D \textbf{39}, 693 (1989).

\bibitem{Furey2018}
C.~Furey, 
``$SU(3)_C \times SU(2)_L \times U(1)_Y (\times U(1)_X)$ as a symmetry of 
division algebraic ladder operators,'' 
Eur.\ Phys.\ J.\ C \textbf{78}, 375 (2018).

\bibitem{Gresnigt2023}
N.~G.~Gresnigt, 
``Three generations of colored fermions with $S_3$ family symmetry from 
Cayley--Dickson sedenions,'' 
Eur.\ Phys.\ J.\ C \textbf{83}, 747 (2023).

\bibitem{Baez2012}
J.~C.~Baez, 
``Division Algebras and Quantum Theory,'' 
Found.\ Phys.\ \textbf{42}, 819 (2012).

\bibitem{WilsonKogut1974}
K.~G.~Wilson and J.~Kogut,
``The renormalization group and the $\epsilon$ expansion,''
Phys.\ Rep.\ \textbf{12}, 75 (1974).

\bibitem{Polchinski1984}
J.~Polchinski,
``Renormalization and effective Lagrangians,''
Nucl.\ Phys.\ B \textbf{231}, 269 (1984).

\bibitem{CallanSymanzik}
C.~G.~Callan,
``Broken scale invariance in scalar field theory,''
Phys.\ Rev.\ D \textbf{2}, 1541 (1970);
K.~Symanzik,
``Small distance behaviour in field theory and power counting,''
Commun.\ Math.\ Phys.\ \textbf{18}, 227 (1970).

\end{thebibliography}

\end{document}
