\documentclass[11pt]{letter}
\usepackage[margin=1in]{geometry}
\usepackage{hyperref}

\signature{[Author Name]}
\address{[Your Institution]\\[Your Address]}

\begin{document}

\begin{letter}{Editorial Board\\Foundations of Physics\\Springer}

\opening{Dear Editors,}

I am pleased to submit the manuscript entitled \textbf{``Realisability Constraints 
and the Emergence of Standard Model Structure''} for consideration for publication 
in \emph{Foundations of Physics}.

\paragraph{Summary.}
The paper presents a \emph{constraint-based analysis} of Standard Model structure. 
Rather than proposing a new dynamical theory, we trace a derivation chain from 
general realisability requirements to specific features of particle physics.

Three contributions distinguish this work:
\begin{enumerate}
\item \textbf{Logical unification}: We connect results from quantum reconstruction 
      theorems (Hardy, CDP), gauge theory (Coleman--Mandula), anomaly cancellation 
      (Geng--Marshak), and division algebra approaches (Furey, Gresnigt) into a 
      single chain with explicit dependencies.
\item \textbf{Saturation theorem}: We prove that the Standard Model \emph{saturates} 
      the space of realisable theories---extensions violate realisability axioms or 
      are merely effective, not fundamental.
\item \textbf{Fundamental/effective criterion}: We distinguish UV-complete structure 
      (constrained by realisability) from emergent effective symmetries (not constrained).
\end{enumerate}

\noindent\textbf{What this paper is not.} This is not a theory of everything. It makes 
no new particle predictions. It does not explain mass values or mixing angles. 
The contribution is \emph{organisational}: connecting known results with explicit 
logical dependencies.

\paragraph{Fit with Foundations of Physics.}
The paper is methodological rather than phenomenological: it does not predict new 
particles or propose experiments. Instead, it asks \emph{why} the mathematical 
structure of fundamental physics takes the form it does. This places it squarely 
within the tradition of foundational analysis that \emph{Foundations of Physics} 
has long championed---from axiomatic quantum mechanics to information-theoretic 
reconstructions.

The manuscript has not been submitted elsewhere and contains no material previously 
published.

\paragraph{Suggested Reviewers.}
\begin{itemize}
\item \textbf{Markus M\"uller} (Institute for Quantum Optics and Quantum Information, 
      Vienna) --- expert on quantum reconstructions and operational approaches.
\item \textbf{Latham Boyle} (Perimeter Institute) --- work on algebraic approaches 
      to the Standard Model.
\item \textbf{Niels Gresnigt} (Xi'an Jiaotong-Liverpool University) --- division 
      algebra approaches to particle physics.
\item \textbf{Fedele Lizzi} (Universit\`a di Napoli) --- noncommutative geometry 
      and particle physics.
\end{itemize}

I believe this work will be of interest to your readership and look forward to 
your consideration.

\closing{Sincerely,}

\end{letter}
\end{document}
