\documentclass[11pt,a4paper]{article}

\usepackage[utf8]{inputenc}
\usepackage[T1]{fontenc}
\usepackage{amsmath,amssymb,amsthm}
\usepackage[colorlinks=true,linkcolor=blue,citecolor=blue]{hyperref}
\usepackage{geometry}
\geometry{margin=1in}

\newtheorem{axiom}{Axiom}
\newtheorem{definition}[axiom]{Definition}
\newtheorem{proposition}[axiom]{Proposition}

\title{\textbf{The Realisability Axiom}\\[0.3em]
\large A One-Page Foundation for Distinction Dynamics}
\author{[Author]}
\date{\today}

\begin{document}
\maketitle

\begin{abstract}
We state the single axiom underlying Distinction Dynamics (DD) and its 
immediate consequences. This note serves as the canonical entry point 
for the DD research program.
\end{abstract}

% ===================================================================
\section*{The Axiom}

\begin{axiom}[Realisability]
\label{ax:realisability}
A structure is \textbf{physically realisable} if and only if it admits 
a faithful embedding into a unitary, information-preserving process algebra.
\end{axiom}

\textbf{Unpacking the terms:}
\begin{itemize}
\item \emph{Faithful embedding}: the structure's relations are preserved, 
      not merely mapped
\item \emph{Unitary}: every process has an inverse; no information is 
      created or destroyed
\item \emph{Information-preserving}: distinguishable inputs yield 
      distinguishable outputs
\item \emph{Process algebra}: composition of processes is associative 
      with identity
\end{itemize}

The axiom encodes one intuition: \emph{to exist physically is to be 
distinguishable, and distinctions must be stable under composition}.

% ===================================================================
\section*{Why Quantum Mechanics (Not Just Any GPT)}

General Probabilistic Theories (GPTs) permit structures beyond quantum 
mechanics. Why does DD select QM specifically?

\begin{proposition}[Hilbert Space from Realisability]
A GPT satisfying:
\begin{enumerate}
\item Purification (every mixed state arises from a pure state of a larger system)
\item Invertibility of pure transformations
\item Continuous reversibility (no discrete jumps in state space)
\end{enumerate}
is necessarily described by a complex Hilbert space.
\end{proposition}

\textit{Sketch.} Conditions (1)--(3) are precisely the realisability 
requirements applied to state spaces. Chiribella--D'Ariano--Perinotti (2011) 
and Masanes--M\"uller (2011) prove that these conditions uniquely select 
quantum theory. Non-Hilbert GPTs violate at least one condition---typically 
invertibility of transformations or continuous composition.

DD inherits quantum structure not by postulate but by consequence.

% ===================================================================
\section*{The Philosophical Core}

\begin{quote}
\textbf{Existence is not the presence of entities, but the closure of 
distinctions under realisation.}
\end{quote}

This positions DD as:
\begin{itemize}
\item Not metaphysics (no claims about ``what there is'' prior to distinction)
\item Not physics (no dynamical equations or predictions of trajectories)
\item An \textbf{ontology of constraints}: what structures can exist given 
      that existence requires distinguishability
\end{itemize}

DD answers: ``Why this structure?'' not ``What happens next?''

% ===================================================================
\section*{Scope and Limitations}

DD is a \textbf{constraint theory}, not a dynamical theory. It does not:

\begin{center}
\begin{tabular}{ll}
\textbf{DD does not explain} & \textbf{Because} \\
\hline
Numerical coupling constants & Requires dynamics (RG flow, matching) \\
Specific mass values & Requires Yukawa textures, not just structure \\
Vacuum selection & Requires cosmological initial conditions \\
Spacetime dimensionality & Separate constraint (future work) \\
Quantum gravity & Beyond current scope \\
\end{tabular}
\end{center}

These are not failures but \textbf{scope boundaries}. DD constrains the 
space of possible theories; dynamics selects trajectories within that space.

% ===================================================================
\section*{What DD Does Derive}

From the Realisability Axiom alone:
\begin{enumerate}
\item \textbf{Quantum structure}: Hilbert space, Born rule, unitarity
\item \textbf{Division algebras}: $\mathbb{R}, \mathbb{C}, \mathbb{H}, \mathbb{O}$ 
      as the only carrier structures (Hurwitz)
\item \textbf{Gauge group}: $SU(3) \times SU(2) \times U(1)$ from octonionic 
      automorphisms
\item \textbf{Hypercharge quantisation}: unique assignment via anomaly 
      cancellation (Geng--Marshak)
\item \textbf{Three generations}: lower bound from CP violation, upper bound 
      from sedenion structure
\item \textbf{Mass hierarchy}: monotonicity with distinction depth (ordering, 
      not values)
\end{enumerate}

The Standard Model emerges as the \emph{unique} realisable structure---not 
the simplest, but the only one.

% ===================================================================
\section*{Entry Points}

\begin{itemize}
\item \textbf{Main paper}: ``Realisability Constraints and the Emergence of 
      Standard Model Structure''
\item \textbf{Companion}: ``Distinction Dynamics as a Constraint Theory''
\item \textbf{Background}: Chiribella et al.\ (2011), Baez (2012), Furey (2018), 
      Gresnigt (2023)
\end{itemize}

\end{document}
