%==============================================================================
% GENERAL RELATIVITY FROM RICCI FLOW
% Einstein Equations as Information Geometry
% Part III, Chapter 5
%==============================================================================

\chapter{General Relativity from Ricci Flow}\label{ch:gr-ricci}

\epigraph{Gravity is the curvature of distinction space.}{---}

\vnew{This chapter is new in DD v2.0. Derives GR from information geometry.}

\section{The Connection}

We established in Chapter~\ref{ch:info-ricci} that:
\begin{equation}
\partial_t g_{ij} = -2\, \mathrm{Ric}_{ij} + 2\, \nabla_i \nabla_j \log p
\end{equation}

Now we show how this connects to Einstein's equations.

\section{Ricci Flow and Einstein Equations}

\subsection{Static Case}

\begin{theorem}[Ricci-Flat $\Leftrightarrow$ Vacuum Einstein]
A manifold satisfies the vacuum Einstein equations:
\begin{equation}
R_{ij} = 0
\end{equation}
if and only if it is a fixed point of the Ricci flow:
\begin{equation}
\partial_t g_{ij} = -2 R_{ij} = 0
\end{equation}
\end{theorem}

\textbf{Interpretation:} Vacuum spacetime = equilibrium of distinction flow.

\subsection{With Matter}

The Einstein equations with matter:
\begin{equation}
R_{ij} - \frac{1}{2}R g_{ij} + \Lambda g_{ij} = 8\pi G\, T_{ij}
\end{equation}

In Ricci flow with source:
\begin{equation}
\partial_t g_{ij} = -2 R_{ij} + \lambda g_{ij} + S_{ij}
\end{equation}

where $S_{ij}$ is a source term and $\lambda$ is a scaling parameter.

\section{The Padmanabhan Program}

T. Padmanabhan (2010, 2015) showed that:

\begin{theorem}[Emergent Gravity]
Einstein's equations can be derived from thermodynamic/information-theoretic principles:
\begin{equation}
\Delta V = \frac{1}{H}(N_{\text{sur}} - N_{\text{bulk}})
\end{equation}
where:
\begin{itemize}
    \item $\Delta V$ = change in spatial volume
    \item $N_{\text{sur}}$ = degrees of freedom on surface
    \item $N_{\text{bulk}}$ = degrees of freedom in bulk
\end{itemize}
\end{theorem}

This is \textbf{holographic equipartition} — the universe expands to equilibrate surface and bulk information.

\section{DD Synthesis}

\subsection{Information $\to$ Curvature}

From Chapter~\ref{ch:info-ricci}:
\begin{equation}
\mathrm{Ric}_{ij} = \frac{1}{2}\partial_t g_{ij} + \nabla_i \nabla_j \log p
\end{equation}

\textbf{DD interpretation:}
\begin{itemize}
    \item $\mathrm{Ric}_{ij}$ = curvature = interaction of distinctions
    \item $\nabla_i \nabla_j \log p$ = information gradient = distinguishability field
\end{itemize}

\subsection{Matter as Distinction Density}

\begin{definition}[Matter-Distinction Correspondence]
The stress-energy tensor corresponds to distinction density:
\begin{equation}
T_{ij} \sim \nabla_i \nabla_j \log p - \frac{1}{2}g_{ij} |\nabla \log p|^2
\end{equation}
\end{definition}

\textbf{Physical meaning:} Mass-energy = concentration of distinguishability.

\section{Derivation Sketch}

\begin{theorem}[Einstein from Fisher-Ricci]
Starting from:
\begin{enumerate}
    \item Fisher metric $g_{ij}$ on statistical manifold
    \item Ricci flow $\partial_t g = -2 \mathrm{Ric}$
    \item Information source $\nabla\nabla \log p$
\end{enumerate}

At equilibrium ($\partial_t g = 0$) with source:
\begin{equation}
\mathrm{Ric}_{ij} = \nabla_i \nabla_j \log p
\end{equation}

Tracing:
\begin{equation}
R = \Delta \log p
\end{equation}

Combining:
\begin{equation}
R_{ij} - \frac{1}{2}R g_{ij} = \nabla_i \nabla_j \log p - \frac{1}{2}g_{ij}\Delta\log p
\end{equation}

This has the \textbf{form of Einstein equations} with the RHS as an effective stress-energy tensor.
\end{theorem}

\section{Cosmological Constant}

\begin{theorem}[Dynamic $\Lambda$]
In DDCE, the cosmological ``constant'' is:
\begin{equation}
\Lambda_{\text{eff}} = k(\Delta + F + M)
\end{equation}

This is \textbf{not constant} — it grows with distinction complexity.
\end{theorem}

\textbf{Connection to DESI observations:} $w(z) \neq -1$ indicates evolving dark energy, consistent with DDCE prediction.

\section{The Complete Picture}

\begin{center}
\begin{tabular}{@{}lll@{}}
\toprule
\textbf{Standard GR} & \textbf{DD/Information} & \textbf{Interpretation} \\
\midrule
Metric $g_{ij}$ & Fisher metric & Distinguishability of states \\
Curvature $R_{ij}$ & Ricci tensor & Interaction of distinctions \\
Matter $T_{ij}$ & $\nabla\nabla\log p$ & Distinction density \\
$\Lambda$ & $k(\Delta+F+M)$ & Total distinction complexity \\
Geodesic & Info-geodesic & Path of minimal distinction \\
\bottomrule
\end{tabular}
\end{center}

\section{Key Equations}

\subsection{Vacuum}
\begin{equation}
R_{ij} = 0 \quad\Leftrightarrow\quad \text{Fixed point of Ricci flow}
\end{equation}

\subsection{With Matter}
\begin{equation}
G_{ij} + \Lambda g_{ij} = 8\pi G \left(\nabla_i \nabla_j \log p - \frac{1}{2}g_{ij}|\nabla\log p|^2\right)
\end{equation}

\subsection{Dynamic (DDCE)}
\begin{equation}
\partial_t g_{ij} = -2 R_{ij} + 2\nabla_i\nabla_j\log p + k(\Delta + F + M)g_{ij}
\end{equation}

\section{Summary}

\begin{center}
\fbox{\parbox{0.9\textwidth}{
\textbf{General relativity is the equilibrium dynamics of distinction geometry.}

\begin{itemize}
    \item Spacetime metric = Fisher metric of physical states
    \item Curvature = interaction between distinctions
    \item Matter = concentration of distinguishability
    \item Expansion = growth of distinction complexity
\end{itemize}

Einstein's equations are the \textbf{fixed-point condition} of Ricci flow with informational source.
}}
\end{center}

\section{Literature}

\begin{itemize}
    \item Padmanabhan, T. (2010). Thermodynamical aspects of gravity. \textit{Rep. Prog. Phys.}
    \item Verlinde, E. (2011). On the origin of gravity. \textit{JHEP}.
    \item Jacobson, T. (1995). Thermodynamics of spacetime. \textit{Phys. Rev. Lett.}
\end{itemize}

DD synthesizes these approaches through \textbf{distinction dynamics}.
