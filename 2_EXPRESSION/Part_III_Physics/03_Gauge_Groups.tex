%==============================================================================
% ELECTROWEAK GAUGE GROUP
% SU(2)_L × U(1)_Y from DD Principles
% Part III, Chapter 3
%==============================================================================

\chapter{Electroweak Gauge Group}\label{ch:gauge-groups}

\epigraph{Chirality is the trace of the triadic direction.}{---}

\vnew{This chapter derives the electroweak structure from DD.}

\section{The Problem}

We derived $SU(3)_C$ for the strong force. Now:

\textbf{Why $SU(2)_L \times U(1)_Y$ for electroweak?}

\section{Chirality from Triadic Direction}

\subsection{The Argument}

From Chapter~\ref{ch:chirality}:
\begin{itemize}
    \item The triad has an intrinsic direction: $A \to B \to C \to A$
    \item This direction is non-eliminable (holonomy)
    \item Physical manifestation: chirality (left vs right)
\end{itemize}

\begin{theorem}[Chirality Necessity]
A triadic system necessarily distinguishes between two orientations.
\end{theorem}

\begin{proof}
\begin{enumerate}
    \item Cycle $A \to B \to C \to A$ has orientation
    \item Reverse cycle $A \to C \to B \to A$ is distinct
    \item No continuous transformation connects them
    \item Therefore: two chiralities exist necessarily
\end{enumerate}
\end{proof}

\subsection{Physical Chirality}

In the Standard Model:
\begin{itemize}
    \item Left-handed fermions: $SU(2)_L$ doublets
    \item Right-handed fermions: $SU(2)_L$ singlets
\end{itemize}

This asymmetry is \textbf{not arbitrary} — it follows from triadic direction.

\section{Why SU(2) for Weak Isospin}

\subsection{Rank Considerations}

\begin{itemize}
    \item Strong: rank 2 (triadic, independent dynamics)
    \item Weak: rank 1 (chiral projection of triad)
\end{itemize}

\begin{theorem}[Weak Isospin Rank]
The chiral projection of a triadic structure has rank 1.
\end{theorem}

\begin{proof}[Argument]
\begin{enumerate}
    \item Full triad: 3 elements, rank 2
    \item Chiral projection: selects one orientation
    \item Projected structure: doublet (2 elements)
    \item Doublet: rank 1
\end{enumerate}
\end{proof}

Therefore: weak isospin is $SU(2)$, not $SU(3)$.

\subsection{Left-Handed Coupling}

\begin{theorem}[Parity Violation]
Weak interactions couple only to left-handed fermions because chirality projection selects one orientation.
\end{theorem}

This is observed experimentally (Wu experiment, 1957).

DD explains \textbf{why} parity is violated — it's the trace of triadic direction.

\section{Why U(1) for Hypercharge}

\subsection{Phase Freedom}

After fixing:
\begin{itemize}
    \item $SU(3)_C$ (color)
    \item $SU(2)_L$ (weak isospin)
\end{itemize}

There remains a residual $U(1)$ phase freedom.

\begin{theorem}[Hypercharge Emergence]
The $U(1)_Y$ hypercharge is the remaining phase after triadic and chiral projections.
\end{theorem}

\subsection{Anomaly Constraints on Hypercharge}

For anomaly cancellation:
\begin{equation}
\sum_f Y_f = 0 \quad \text{(per generation)}
\end{equation}

This constrains hypercharge assignments:
\begin{center}
\begin{tabular}{@{}lcc@{}}
\toprule
\textbf{Particle} & $SU(2)_L$ & $Y$ \\
\midrule
$Q_L = (u_L, d_L)$ & $\mathbf{2}$ & $+1/6$ \\
$u_R$ & $\mathbf{1}$ & $+2/3$ \\
$d_R$ & $\mathbf{1}$ & $-1/3$ \\
$L = (\nu_L, e_L)$ & $\mathbf{2}$ & $-1/2$ \\
$e_R$ & $\mathbf{1}$ & $-1$ \\
\bottomrule
\end{tabular}
\end{center}

Check: $3 \times 2 \times \frac{1}{6} + 3 \times \frac{2}{3} + 3 \times (-\frac{1}{3}) + 2 \times (-\frac{1}{2}) + (-1) = 1 + 2 - 1 - 1 - 1 = 0$ ✓

\section{The Full Standard Model Gauge Group}

\begin{equation}
G_{SM} = SU(3)_C \times SU(2)_L \times U(1)_Y
\end{equation}

\subsection{DD Derivation Summary}

\begin{center}
\begin{tabular}{@{}lll@{}}
\toprule
\textbf{Group} & \textbf{DD Origin} & \textbf{Constraint} \\
\midrule
$SU(3)_C$ & Triadic necessity & Rank $\geq 2$, anomaly-free \\
$SU(2)_L$ & Chiral projection & Rank 1, left-handed \\
$U(1)_Y$ & Residual phase & Anomaly cancellation \\
\bottomrule
\end{tabular}
\end{center}

\section{Electroweak Symmetry Breaking}

\subsection{The Higgs Mechanism}

At low energies:
\begin{equation}
SU(2)_L \times U(1)_Y \xrightarrow{\langle H \rangle} U(1)_{EM}
\end{equation}

\begin{theorem}[Symmetry Breaking Necessity]
Spontaneous symmetry breaking is required for:
\begin{enumerate}
    \item Massive $W^\pm$, $Z$ bosons
    \item Massless photon
    \item Fermion masses via Yukawa couplings
\end{enumerate}
\end{theorem}

\subsection{DD Interpretation}

Symmetry breaking = \textbf{distinction crystallization}.

At high energy: all distinctions equivalent (symmetric).

At low energy: specific distinctions become stable (broken).

The Higgs field $H$ is the \textbf{order parameter} of distinction crystallization.

\section{Three Generations}

\subsection{The Mystery}

Why exactly 3 generations of fermions?

\subsection{DD Argument}

\begin{theorem}[Three Generations from Triad]
The number of fermion generations equals the number of elements in the minimal triadic structure.
\end{theorem}

\begin{proof}[Argument]
\begin{enumerate}
    \item Triad: $(A, B, C)$ — three elements
    \item Each element corresponds to one generation
    \item Generations are ``flavors'' of the same triadic structure
    \item Therefore: 3 generations
\end{enumerate}
\end{proof}

This is consistent with:
\begin{itemize}
    \item LEP measurement: $N_\nu = 2.984 \pm 0.008$
    \item CKM/PMNS mixing matrices are $3 \times 3$
\end{itemize}

\section{CP Violation}

\subsection{Complex Phases}

The CKM matrix has one physical CP-violating phase.

\begin{theorem}[CP Violation from Triadic Phases]
CP violation arises from non-eliminable phases in the triadic structure.
\end{theorem}

Recall from Chapter~\ref{ch:dyad}:
\begin{itemize}
    \item SU(2): all phases eliminable (rank 1)
    \item SU(3): two independent phases (rank 2)
    \item One phase becomes CP-violating
\end{itemize}

\section{The Complete Picture}

\begin{center}
\fbox{\parbox{0.95\textwidth}{
\textbf{Standard Model Gauge Structure from DD:}

\begin{enumerate}
    \item \textbf{Existence} ($\Delta \neq \emptyset$) $\Rightarrow$ Something
    \item \textbf{Triadic necessity} (rank $\geq 2$) $\Rightarrow$ $SU(3)_C$
    \item \textbf{Chirality} (triadic direction) $\Rightarrow$ $SU(2)_L$ (left-handed)
    \item \textbf{Phase freedom} $\Rightarrow$ $U(1)_Y$
    \item \textbf{Anomaly cancellation} $\Rightarrow$ Hypercharge assignments
    \item \textbf{Triadic count} $\Rightarrow$ 3 generations
    \item \textbf{Non-eliminable phases} $\Rightarrow$ CP violation
\end{enumerate}

The Standard Model is \textbf{not arbitrary}. It is the unique self-consistent realization of distinction dynamics.
}}
\end{center}

\section{What DD Does Not (Yet) Derive}

\begin{itemize}
    \item Specific values of coupling constants
    \item Fermion mass ratios
    \item Higgs mass
    \item Neutrino masses and mixing
\end{itemize}

These may require additional structure (see Chapter~\ref{ch:constants}).

\section{Summary}

\begin{theorem}[Standard Model Uniqueness]
The gauge group $SU(3)_C \times SU(2)_L \times U(1)_Y$ with 3 generations is the unique anomaly-free, asymptotically free, confining, chiral gauge theory compatible with triadic necessity.
\end{theorem}

There is no choice. There is no legislator. There is only consistency.
