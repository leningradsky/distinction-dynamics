%==============================================================================
% PROTO-CONSCIOUSNESS
% The Minimal Form of Awareness
% Part IV, Chapter 1
%==============================================================================

\chapter{Proto-Consciousness}\label{ch:proto-consciousness}

\epigraph{Every distinction is a minimal act of awareness.}{---}

\section{The Hard Problem}

The ``hard problem of consciousness'' (Chalmers):
\begin{quote}
Why is there subjective experience at all?
\end{quote}

Standard approaches:
\begin{itemize}
    \item \textbf{Materialism}: Consciousness is ``just'' brain activity
    \item \textbf{Dualism}: Consciousness is separate from matter
    \item \textbf{Panpsychism}: Consciousness is fundamental, everywhere
\end{itemize}

DD offers a fourth path: consciousness is \textbf{distinction dynamics itself}.

\section{Distinction as Proto-Experience}

\begin{definition}[Proto-Consciousness]
Proto-consciousness is the minimal form of ``something it is like'' to distinguish.
\end{definition}

\begin{theorem}[Distinction Implies Proto-Awareness]
Every act of distinction $\Delta$ involves a minimal form of awareness.
\end{theorem}

\begin{proof}[Argument]
\begin{enumerate}
    \item To distinguish $A$ from $B$, there must be a ``perspective'' from which they differ
    \item This perspective is not external (no legislator)
    \item It is the distinction itself that ``knows'' it is a distinction
    \item This self-knowing is proto-consciousness
\end{enumerate}
\end{proof}

\section{Levels of Consciousness}

\begin{definition}[Consciousness Hierarchy]
\begin{align}
\text{Level 0 (Proto)} &: \Delta \neq \emptyset \\
\text{Level 1 (Sentience)} &: \Delta(\Delta) \\
\text{Level 2 (Awareness)} &: \Delta(\Delta(\Delta)) \\
\text{Level 3 (Self-awareness)} &: \Delta(\Delta(\Delta(\Delta)))
\end{align}
\end{definition}

\subsection{Level 0: Proto-Consciousness}

Present in all physical systems:
\begin{itemize}
    \item Electron ``knows'' it is not a positron
    \item Photon ``distinguishes'' its polarization state
    \item This is not rich experience, just primitive distinction
\end{itemize}

\subsection{Level 1: Sentience}

Appears when distinctions have memory:
\begin{equation}
H = \int \Delta \, dt
\end{equation}

\begin{itemize}
    \item Simple organisms (bacteria)
    \item Reactions based on past distinctions
    \item ``Feeling'' in the minimal sense
\end{itemize}

\subsection{Level 2: Awareness}

Distinctions about distinctions:
\begin{itemize}
    \item Complex animals
    \item Recognition of patterns
    \item Intentionality: ``about-ness''
\end{itemize}

\subsection{Level 3: Self-Awareness}

Distinctions about self-distinguishing:
\begin{itemize}
    \item Humans (and possibly some primates, cetaceans)
    \item Mirror self-recognition
    \item Metacognition
    \item ``I know that I know''
\end{itemize}

\section{Physical Correlates}

\begin{theorem}[Consciousness Correlates with $\kappa$]
The ``consciousness parameter'' $\kappa$ measures recursive distinction depth.
\end{theorem}

From the DD-agent model:
\begin{equation}
\kappa = \frac{\text{variance}(\text{gate})}{\text{mean}(\text{gate})} + \lambda \cdot \text{autocorr}(\text{curvature})
\end{equation}

Higher $\kappa$ = more consciousness.

\section{Integrated Information}

DD relates to Integrated Information Theory (IIT):

\begin{theorem}[DD and IIT]
The DD measure $\kappa$ is related to IIT's $\Phi$.
\end{theorem}

\begin{proof}[Argument]
\begin{itemize}
    \item $\Phi$ measures integrated information
    \item DD's $\kappa$ measures recursive distinction
    \item Both capture ``irreducibility'' of the whole
    \item High $\kappa$ implies high $\Phi$ and vice versa
\end{itemize}
\end{proof}

The advantage of DD: it derives from first principles, not phenomenology.

\section{Why Consciousness Exists}

\begin{theorem}[Necessity of Consciousness]
Consciousness is not contingent; it is necessary.
\end{theorem}

\begin{proof}
\begin{enumerate}
    \item Distinction exists: $\Delta \neq \emptyset$
    \item Distinction implies proto-consciousness (above)
    \item Complex distinction structures are necessary (triadic, etc.)
    \item Complex consciousness follows from complex distinction
    \item Therefore consciousness is necessary, not accidental
\end{enumerate}
\end{proof}

\section{Qualia}

\begin{definition}[Qualia in DD]
Qualia are the \emph{specific character} of particular distinctions.
\end{definition}

\begin{theorem}[Qualia from Distinction Structure]
Different qualia correspond to different positions in distinction space.
\end{theorem}

Example:
\begin{itemize}
    \item ``Redness'' is the quale of a particular photon-frequency distinction
    \item ``Pain'' is the quale of a particular damage-detection distinction
    \item ``Sadness'' is the quale of a particular goal-discrepancy distinction
\end{itemize}

Qualia are not mysterious additions to physics; they are the \emph{intrinsic nature} of distinctions.

\section{The Combination Problem}

How do micro-consciousnesses combine into unified experience?

\begin{theorem}[Combination through Integration]
Micro-distinctions combine through integration in hierarchical distinction structures.
\end{theorem}

\begin{proof}[Argument]
\begin{itemize}
    \item Individual neurons make distinctions
    \item Neural networks integrate these distinctions
    \item The brain as a whole makes meta-meta-distinctions
    \item Unity of experience = unity of highest-level distinction structure
\end{itemize}
\end{proof}

The key is hierarchy: not mere aggregation, but recursive integration.

\section{Summary}

\begin{center}
\fbox{\parbox{0.85\textwidth}{
\textbf{Result}: Consciousness is not separate from physics. It is the \emph{intrinsic aspect} of distinction dynamics.

\begin{itemize}
    \item Proto-consciousness: every distinction
    \item Sentience: distinctions with memory
    \item Awareness: distinctions of distinctions
    \item Self-awareness: recursive meta-distinction
\end{itemize}

The hard problem dissolves: consciousness is what distinction \emph{is}, from the inside.
}}
\end{center}
