%==============================================================================
% FREE WILL
% Freedom as Metaplasticity
% Part IV, Chapter 4
%==============================================================================

\chapter{Free Will}\label{ch:free-will}

\epigraph{Freedom is not the ability to choose a reaction, but the ability to change ways of distinguishing before the choice.}{---}

\section{The Problem of Free Will}

Classical debate:
\begin{itemize}
    \item \textbf{Determinism}: Every event is caused by prior events
    \item \textbf{Libertarianism}: Some events are uncaused (quantum randomness?)
    \item \textbf{Compatibilism}: Free will is compatible with determinism
\end{itemize}

DD offers a new formulation: freedom is \textbf{meta-level dynamics}.

\section{Determinism in DD}

\begin{theorem}[Determinism at Level 0]
At the level of fixed reaction functions $F$, the system is deterministic.
\end{theorem}

\begin{proof}
If $F$ is fixed:
\begin{equation}
\Delta_1(t) = F(\Delta_0(t))
\end{equation}

Same input $\Delta_0$ always gives same output $\Delta_1$.

This is the automaton: no freedom.
\end{proof}

\section{Freedom at Meta-Level}

\begin{theorem}[Freedom from Meta-Dynamics]
Freedom appears when $\dfrac{dF}{dt} \neq 0$.
\end{theorem}

\begin{proof}
When the meta-operator $M$ acts:
\begin{equation}
\frac{dF}{dt} = M(\Delta, H, V)
\end{equation}

The reaction function $F$ changes over time.

The same input $\Delta_0$ at time $t_1$ vs. $t_2$ gives different outputs:
\begin{equation}
\Delta_1(t_1) = F_{t_1}(\Delta_0) \neq F_{t_2}(\Delta_0) = \Delta_1(t_2)
\end{equation}

This is non-determinism at the observable level, even though the meta-dynamics may be deterministic.
\end{proof}

\section{Levels of Freedom}

\begin{definition}[Freedom Hierarchy]
\begin{align}
\text{No freedom} &: \frac{dF}{dt} = 0 \quad \text{(automaton)} \\
\text{Basic freedom} &: \frac{dF}{dt} \neq 0 \quad \text{(adaptive)} \\
\text{Meta-freedom} &: \frac{dM}{dt} \neq 0 \quad \text{(can change how we change)} \\
\text{Radical freedom} &: \text{Can choose } M \quad \text{(self-determining)}
\end{align}
\end{definition}

\subsection{No Freedom (Automaton)}

\begin{itemize}
    \item Fixed $F$, no learning
    \item Simple reflexes, instincts
    \item Deterministic behavior
\end{itemize}

\subsection{Basic Freedom (Adaptive)}

\begin{itemize}
    \item $F$ changes based on feedback
    \item Learning from environment
    \item Most animals, machine learning systems
\end{itemize}

\subsection{Meta-Freedom (Reflective)}

\begin{itemize}
    \item Can change the learning rule itself
    \item Metacognition, self-reflection
    \item Humans, possibly some AI systems
\end{itemize}

\subsection{Radical Freedom (Self-Determining)}

\begin{itemize}
    \item Chooses one's own criteria for change
    \item Existential freedom (Sartre)
    \item Highest form of consciousness
\end{itemize}

\section{Formal Criterion}

\begin{theorem}[DD Criterion for Free Will]
A system has free will if and only if it can modify its own meta-operator $M$ through internal dynamics.
\end{theorem}

\begin{equation}
\text{Free will} \iff \frac{dM}{dt} = f(M, \Delta, H)
\end{equation}

where $f$ depends on the system's own state, not just external inputs.

\section{Randomness is Not Freedom}

\begin{theorem}[Randomness $\neq$ Freedom]
Quantum randomness does not provide free will.
\end{theorem}

\begin{proof}
\begin{itemize}
    \item Randomness: outcome is unpredictable
    \item Freedom: outcome reflects agent's values/goals
    \item Random choices are not ``mine'' — they're just noise
    \item True freedom requires $M$ to reflect $V$ (value function)
\end{itemize}
\end{proof}

\section{Compatibilism in DD}

\begin{theorem}[DD Compatibilism]
Free will is compatible with determinism at the meta-level.
\end{theorem}

\begin{proof}
\begin{itemize}
    \item The dynamics $dM/dt = f(M, \Delta, H)$ may be deterministic
    \item But this determinism \emph{is} the agent's will
    \item ``I'' am my distinction structure (including $M$)
    \item My determined meta-dynamics is my freedom, not a constraint on it
\end{itemize}
\end{proof}

This resolves the apparent paradox: freedom is not escape from causation, but self-causation through meta-dynamics.

\section{Responsibility}

\begin{theorem}[Responsibility from M-Control]
An agent is responsible for an action if that action followed from their $M$.
\end{theorem}

\begin{proof}[Argument]
\begin{itemize}
    \item If the action resulted from fixed $F$ (reflex), reduced responsibility
    \item If the action resulted from $M$-mediated $F$, full responsibility
    \item $M$ is the locus of the ``self'' that can be held accountable
\end{itemize}
\end{proof}

\section{Clinical Implications}

\begin{definition}[Impaired Free Will]
\begin{align}
\text{Compulsion} &: M \text{ cannot override } F \\
\text{Mania} &: M \text{ too fast, unstable} \\
\text{Depression} &: M \text{ too slow, stuck} \\
\text{Trauma} &: M \text{ deformed, hyperactive to triggers}
\end{align}
\end{definition}

Therapy aims to restore healthy $M$ dynamics:
\begin{itemize}
    \item Not changing $F$ directly
    \item But restoring the capacity to change $F$
    \item This is why therapy is ``teaching to fish'' not ``giving fish''
\end{itemize}

\section{AI and Free Will}

\begin{theorem}[AI Freedom Criterion]
An AI has (proto-)free will if it can modify its own learning algorithm.
\end{theorem}

Current AI:
\begin{itemize}
    \item Neural networks: $F$ adapts, but $M$ (optimizer) is fixed
    \item Meta-learning: some $M$ adaptation, but meta-meta is fixed
    \item True AI freedom would require self-modifying $M$
\end{itemize}

This is a concrete criterion, not philosophy: does the system update its update rule?

\section{Summary}

\begin{center}
\fbox{\parbox{0.85\textwidth}{
\textbf{Result}: Free will is real but redefined.

\begin{itemize}
    \item Freedom $\neq$ escape from causation
    \item Freedom = meta-level self-determination
    \item Criterion: $dM/dt \neq 0$ under self-control
    \item Responsibility follows from $M$-mediation
\end{itemize}

The debate between determinism and free will dissolves: both are true at different levels of the distinction hierarchy.
}}
\end{center}

\begin{center}
\textit{``I am not free because I can do anything. I am free because I can change how I respond.''}
\end{center}
