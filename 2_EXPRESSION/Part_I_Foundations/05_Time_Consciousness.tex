%==============================================================================
% Chapter: Time and Proto-Consciousness
% Part I: Foundations
%==============================================================================

\chapter{Time and Proto-Consciousness}\label{ch:time-consc}

\epigraph{$\Self$ is both the arrow and the eye.}{---}

The structure $\Self$ inherently contains \textbf{time} and \textbf{proto-consciousness}. These are not additions---they are aspects of the axiom itself.

\section{Time as Emergent from Recursion}

\begin{theorem}[Necessity of Time]\label{thm:time}
Time exists necessarily.
\end{theorem}

\begin{proof}
\begin{enumerate}
    \item $\Self$ (Axiom~\ref{ax:dd}).
    \item The equation is recursive: the right-hand side contains $\Delta$.
    \item Unfolding: $\Delta = \Delta(\Delta) = \Delta(\Delta(\Delta)) = \cdots$
    \item This generates a hierarchy of levels: $\Delta^{(0)}, \Delta^{(1)}, \Delta^{(2)}, \ldots$
    \item The levels are ordered: level $n$ is ``inside'' level $n+1$.
    \item This ordering is a proto-temporal structure: $n$ is ``before'' $n+1$ in the unfolding.
    \item Time is not presupposed---it \emph{emerges} from the recursive structure.
    \item Since $\Self$ is necessary, the unfolding is necessary.
    \item Therefore, time is necessary. \qedhere
\end{enumerate}
\end{proof}

\begin{remark}
We do not assume that $\Delta(\Delta)$ is a ``process in time.'' Rather, the recursive structure of $\Self$ \emph{generates} the order we call time. Time is the trace of self-reference.
\end{remark}

\section{Time from Triadic Phase Structure}

\vnew{This section connects to Chapter~\ref{ch:dyad}.}

\begin{theorem}[Time from Triad]\label{thm:time-triad}
Time emerges from non-eliminable phase differences in the triadic structure.
\end{theorem}

\begin{proof}
\begin{enumerate}
    \item By Theorem~\ref{thm:triad-suff}, the triad is the minimal self-sufficient structure.
    \item The triad has three eigenvalues: $\lambda_1, \lambda_2, \lambda_3$.
    \item Phase evolution: $\phi_{ij}(t) = (\lambda_i - \lambda_j)t$.
    \item Two independent phase differences: $\phi_{12}, \phi_{23}$.
    \item These cannot all be eliminated by global transformation (unlike $SU(2)$).
    \item The non-eliminable phases constitute \emph{intrinsic} time.
    \item Time is not imposed from outside but emerges from triadic structure. \qedhere
\end{enumerate}
\end{proof}

\begin{corollary}
The arrow of time is related to CP-violation in the triadic mixing matrix.
\end{corollary}

\section{Proto-Consciousness as Reflexivity}

\begin{definition}[Proto-Consciousness]\label{def:proto}
\textbf{Proto-consciousness} is the minimal structure of self-reference: a system $S$ such that $S = S(S)$.
\end{definition}

\begin{theorem}[Necessity of Proto-Consciousness]\label{thm:consciousness}
Proto-consciousness exists necessarily.
\end{theorem}

\begin{proof}
\begin{enumerate}
    \item $\Self$ (Axiom~\ref{ax:dd}).
    \item This satisfies Definition~\ref{def:proto}: $\Delta$ is a system such that $\Self$.
    \item Therefore, $\Delta$ is proto-conscious.
    \item $\Delta$ exists necessarily (Corollary~\ref{cor:exists}).
    \item Therefore, proto-consciousness exists necessarily. \qedhere
\end{enumerate}
\end{proof}

\begin{remark}
We distinguish:
\begin{itemize}
    \item \textbf{Reflexivity}: A system responds to itself (e.g., thermostat).
    \item \textbf{Proto-consciousness}: A system \emph{is} its own self-application ($S = S(S)$).
    \item \textbf{Full consciousness}: Rich phenomenal experience (human minds).
\end{itemize}
DD claims proto-consciousness is fundamental. Full consciousness is an elaboration, derived in Part~IV.
\end{remark}

\begin{remark}
$\Self$ is not merely ``distinction affecting itself'' (like a thermostat). It is ``distinction \emph{distinguishing} itself''---modeling its own operation. This is the minimal structure of awareness.
\end{remark}

\section{Duality of Time and Consciousness}

\begin{theorem}[Time-Consciousness Duality]\label{thm:duality}
Time and proto-consciousness are dual aspects of $\Self$.
\end{theorem}

\begin{proof}
\begin{enumerate}
    \item Time emerges from the recursive unfolding of $\Delta = \Delta(\Delta(\cdots))$ (Theorem~\ref{thm:time}).
    \item Proto-consciousness is the self-referential structure $\Self$ (Theorem~\ref{thm:consciousness}).
    \item Both arise from the same equation.
    \item Time emphasizes the \emph{sequential} aspect (levels of unfolding).
    \item Proto-consciousness emphasizes the \emph{reflexive} aspect (self-application).
    \item Sequence and reflexion are two views of one structure. \qedhere
\end{enumerate}
\end{proof}

\begin{corollary}[The Observer]
The universe observes itself. Observation is not external to $\Delta$---it is $\Self$.
\end{corollary}

\section{Why Consciousness and Time Are Linked}

\begin{proposition}
Without time, there is no consciousness. Without consciousness, there is no time.
\end{proposition}

\begin{proof}[Argument]
\begin{enumerate}
    \item Consciousness requires distinguishing states $\Rightarrow$ requires sequence $\Rightarrow$ requires time.
    \item Time requires marking ``before'' vs ``after'' $\Rightarrow$ requires distinction $\Rightarrow$ requires observer-structure $\Rightarrow$ requires proto-consciousness.
    \item They co-arise from $\Self$.
\end{enumerate}
\end{proof}

\begin{remark}
This resolves the ``hard problem of consciousness'' at the foundational level: consciousness is not emergent from non-conscious matter. It is built into the structure of distinction itself. The question ``why is there consciousness?'' has the same answer as ``why is there distinction?''---because its denial is self-contradictory.
\end{remark}
