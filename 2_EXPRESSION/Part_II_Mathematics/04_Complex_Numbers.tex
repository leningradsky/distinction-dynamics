%==============================================================================
% COMPLEX NUMBERS
% Why i² = -1
% Part II, Chapter 4
%==============================================================================

\chapter{Complex Numbers}\label{ch:complex}

\epigraph{$i$ is the minimal closure under self-referential rotation.}{---}

\section{Why Complex Numbers?}

Real numbers are insufficient for quantum mechanics. Why?

DD answer: Self-reference requires rotation, and $\mathbb{C}$ is the minimal closure under rotation.

\section{Rotation from Self-Reference}

\begin{theorem}[Self-Reference Implies Rotation]
The operation $\Delta(\Delta)$ is a rotation.
\end{theorem}

\begin{proof}
\begin{enumerate}
    \item $\Delta$ applied to itself gives $\Delta$
    \item But the ``position'' has changed (outer became inner)
    \item This change of position without change of content is rotation
    \item In the complex plane: multiplication by $e^{i\theta}$
\end{enumerate}
\end{proof}

\section{The Imaginary Unit}

\begin{theorem}[Necessity of $i$]
$i$ with $i^2 = -1$ is the minimal solution to:
\begin{quote}
``What operation, applied twice, gives the opposite?''
\end{quote}
\end{theorem}

\begin{proof}
\begin{itemize}
    \item We need $x$ such that $x \cdot x = -1$
    \item In $\mathbb{R}$: no solution (positive times positive is positive)
    \item Minimal extension: add $i$ with $i^2 = -1$
    \item This gives $\mathbb{C} = \mathbb{R}[i]$
\end{itemize}
\end{proof}

\section{Phase and Distinction}

\begin{theorem}[Phase as Internal Distinction]
The phase $\theta$ in $e^{i\theta}$ measures ``internal orientation'' of a distinction.
\end{theorem}

\begin{proof}[Argument]
\begin{itemize}
    \item Two distinctions may have the same magnitude but different phases
    \item Phase difference is physical (interference)
    \item Phase itself is gauge-dependent
    \item This is exactly the structure of quantum mechanics
\end{itemize}
\end{proof}

\section{Complex Numbers in Quantum Mechanics}

\begin{theorem}[QM Requires $\mathbb{C}$]
Quantum mechanics cannot be formulated over $\mathbb{R}$ alone.
\end{theorem}

Recent proof (Renou et al., 2021): Bell-type experiments distinguish quantum theory from ``real quantum theory.''

DD explanation:
\begin{itemize}
    \item Quantum states are distinction distributions
    \item Distinctions have phase (internal orientation)
    \item Phase requires $\mathbb{C}$
\end{itemize}

\section{Algebraic Closure}

\begin{theorem}[Fundamental Theorem of Algebra]
Every polynomial over $\mathbb{C}$ has a root in $\mathbb{C}$.
\end{theorem}

DD interpretation:
\begin{itemize}
    \item $\mathbb{C}$ is algebraically closed
    \item No further extensions needed
    \item This is why physics uses $\mathbb{C}$, not quaternions or octonions (for most purposes)
\end{itemize}

\section{Higher Algebras}

\begin{table}[h]
\centering
\begin{tabular}{@{}lccc@{}}
\toprule
Algebra & Dimension & Associative? & Commutative? \\
\midrule
$\mathbb{R}$ & 1 & Yes & Yes \\
$\mathbb{C}$ & 2 & Yes & Yes \\
$\mathbb{H}$ (quaternions) & 4 & Yes & No \\
$\mathbb{O}$ (octonions) & 8 & No & No \\
\bottomrule
\end{tabular}
\caption{Division algebras over $\mathbb{R}$}
\end{table}

$\mathbb{C}$ is the minimal extension that:
\begin{enumerate}
    \item Contains solutions to all polynomial equations
    \item Maintains commutativity and associativity
\end{enumerate}

\section{Euler's Identity}

\begin{equation}
e^{i\pi} + 1 = 0
\end{equation}

DD interpretation:
\begin{itemize}
    \item $e$: growth (accumulation of distinctions)
    \item $i$: rotation (self-reference)
    \item $\pi$: half-turn (complete reversal)
    \item $1$: identity (the distinction that is)
    \item $0$: nullity (the distinction that isn't)
\end{itemize}

The identity says: growth under self-referential rotation by a half-turn gives the opposite of identity.

\section{Summary}

\begin{center}
\fbox{\parbox{0.85\textwidth}{
\textbf{Result}: Complex numbers are necessary because self-reference is rotational.

\begin{itemize}
    \item $i^2 = -1$: minimal closure under ``twice = opposite''
    \item Phase: internal orientation of distinction
    \item Quantum mechanics requires $\mathbb{C}$
    \item $\mathbb{C}$ is algebraically closed — minimal complete extension
\end{itemize}
}}
\end{center}
