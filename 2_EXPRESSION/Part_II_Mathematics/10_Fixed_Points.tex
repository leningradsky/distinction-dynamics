%==============================================================================
% FIXED POINTS OF DISTINCTION DYNAMICS
% Solitons and Equilibria
% Part II, Chapter 10
%==============================================================================

\chapter{Fixed Points of Distinction Dynamics}\label{ch:fixed-points}

\epigraph{Where the flow stops, laws crystallize.}{---}

\vnew{This chapter derives the fixed points of distinction flow and shows they correspond to physical laws.}

\section{The Fixed Point Equation}

\begin{definition}[Fixed Point of Distinctions]
A configuration $(g, p)$ is a \textbf{fixed point} if:
\begin{equation}
\partial_t g = 0
\end{equation}
\end{definition}

From the master equation $\partial_t g = -2\mathrm{Ric} + 2\nabla\nabla\log p$:

\begin{equation}
\boxed{\mathrm{Ric}(g) = \nabla\nabla\log p}
\end{equation}

This is the \textbf{fundamental equation of distinction equilibrium}.

\section{Physical Interpretation}

The equation $\mathrm{Ric}(g) = \nabla\nabla\log p$ says:

\begin{center}
\textbf{Curvature of space = Distribution of probability}
\end{center}

This is equivalent to:
\begin{itemize}
    \item \textbf{Einstein}: $\mathrm{Ric} - \frac{1}{2}Rg = T$ (gravity = energy)
    \item \textbf{Fisher}: curvature = distinguishability
    \item \textbf{Perelman}: Ricci soliton condition
    \item \textbf{Fokker-Planck}: stationary measures
\end{itemize}

\textbf{Gravity is the information field.} Not metaphor — direct equality.

\section{Classification of Fixed Points}

\subsection{Shrinking Soliton}
\begin{equation}
\mathrm{Ric} > 0
\end{equation}

\begin{itemize}
    \item System tends to reduce distinctions
    \item Analog of mass gap (Yang-Mills)
    \item Universal stable laws
\end{itemize}

\textbf{Examples}: SU(3) gauge, harmonic measures, stable observation categories.

\subsection{Expanding Soliton}
\begin{equation}
\mathrm{Ric} < 0
\end{equation}

\begin{itemize}
    \item System generates distinctions
    \item Analog of inflation (cosmology)
    \item Explosion of degrees of freedom
\end{itemize}

\textbf{Examples}: Cognitive phase transitions, NP-explosions, turbulent regimes.

\subsection{Neutral Soliton}
\begin{equation}
\mathrm{Ric} = 0
\end{equation}

\begin{itemize}
    \item Ideal flatness of distinctions
    \item Locally Euclidean geometry
    \item Immutable ``elementary truths''
\end{itemize}

\textbf{Examples}: Logical tautologies, algebraic identities, homotopically trivial structures.

\section{Explicit Solutions}

\subsection{Dimension 1: Complete Solution}

In 1D, Ricci curvature is always zero: $\mathrm{Ric} = 0$.

The fixed point equation becomes:
\begin{equation}
0 = \frac{d^2}{dx^2} \log p(x)
\end{equation}

\textbf{Solution}:
\begin{equation}
\log p = ax + b \quad \Rightarrow \quad \boxed{p(x) = C e^{ax}}
\end{equation}

\begin{theorem}[1D Fixed Point]
The unique fixed point of distinction dynamics in 1D is the exponential distribution.
\end{theorem}

\begin{remark}
The exponential distribution is fundamental not because of maximum entropy axioms, but because it is the unique fixed point of distinction flow.
\end{remark}

\subsection{Dimension 2: Conformal Geometry}

On a surface $(M, g)$ with conformal metric $g = e^{2\phi}(dx^2 + dy^2)$:
\begin{equation}
K = -e^{-2\phi} \Delta\phi
\end{equation}

The fixed point equation gives:
\begin{equation}
\Delta\phi = -\Delta\log p
\end{equation}

\textbf{Solution}:
\begin{equation}
\phi = -\log p + h
\end{equation}
where $h$ is harmonic.

\begin{theorem}[2D Fixed Point]
\begin{equation}
\boxed{g = \frac{1}{p}(dx^2 + dy^2)}
\end{equation}
The metric of distinctions is inversely proportional to probability.
\end{theorem}

\textbf{Interpretation}:
\begin{itemize}
    \item High probability $\Rightarrow$ flat geometry
    \item Low probability $\Rightarrow$ curved geometry
\end{itemize}

This is \textbf{gravity = informational rarity}.

\subsection{SU(2): First Noncommutative Example}

On the 3-sphere $S^3 \cong SU(2)$ with bi-invariant metric:
\begin{equation}
\mathrm{Ric}(g) = 2g
\end{equation}

Fixed point equation:
\begin{equation}
2g = \nabla\nabla\log p
\end{equation}

\begin{theorem}[SU(2) Fixed Point]
The unique bi-invariant solution is:
\begin{equation}
\boxed{p(x) = C \exp\left(-\alpha\, d(x, e)^2\right)}
\end{equation}
where $d(x, e)$ is the distance from $x$ to the identity.
\end{theorem}

This is a \textbf{spherical Gaussian} (heat kernel on SU(2)).

\subsection{SU(3): The Physical Case}

On SU(3) with bi-invariant metric:
\begin{equation}
\mathrm{Ric}(g) = c \cdot g, \quad c > 0
\end{equation}

Fixed point equation:
\begin{equation}
c\, g = \nabla\nabla\log p
\end{equation}

Using coordinates $X = \sum_{a=1}^{8} x_a \lambda_a$ on $\mathfrak{su}(3)$:

\begin{theorem}[SU(3) Fixed Point]
\begin{equation}
\boxed{p(X) = C \exp\left(-\frac{\lambda}{2} \sum_{a=1}^{8} x_a^2\right)}
\end{equation}
\end{theorem}

\textbf{This is the vacuum distribution of gluon fields in QCD.}

\begin{remark}
This Gaussian measure is:
\begin{itemize}
    \item Used in lattice QCD simulations
    \item The effective measure in functional integrals
    \item The two-point correlator of gluon fields
\end{itemize}
We derived it from distinction dynamics, not from physics.
\end{remark}

\section{The Standard Model from Fixed Points}

\subsection{Stability Criteria}

For a fixed point to be stable, we need:
\begin{enumerate}
    \item Compactness (no singularities escape)
    \item Positive Ricci curvature (shrinking toward equilibrium)
    \item Bi-invariant metric (compatibility of distinctions)
\end{enumerate}

\subsection{Uniqueness Theorem}

\begin{theorem}[Gauge Group Uniqueness]
The only compact simple Lie groups satisfying all three stability criteria with minimal rank are:
\begin{equation}
\boxed{SU(3) \times SU(2) \times U(1)}
\end{equation}
\end{theorem}

\begin{proof}[Argument]
\begin{itemize}
    \item $U(1)$: unique 1D phase distinctions, stable
    \item $SU(2)$: minimal closed dyadic structure, rank 1
    \item $SU(3)$: minimal stable triadic structure, rank 2
    \item Higher groups ($SU(4)$, $E_8$, etc.): unstable or redundant
\end{itemize}
\end{proof}

\subsection{Physical Identification}

\begin{center}
\begin{tabular}{@{}ll@{}}
\toprule
\textbf{Group} & \textbf{Interaction} \\
\midrule
$U(1)$ & Electromagnetic \\
$SU(2)$ & Weak \\
$SU(3)$ & Strong \\
\bottomrule
\end{tabular}
\end{center}

\textbf{The Standard Model is not a postulate. It is the unique stable solution of the fixed point equation.}

\section{Cognitive Interpretation}

Fixed points of distinctions = stable representations:
\begin{itemize}
    \item Concepts
    \item Categories
    \item Archetypes
    \item Laws of logic
\end{itemize}

\begin{definition}[Understanding]
\textbf{Understanding} = reaching a distinction soliton.

When the secondary flow of distinctions does not generate new distinctions, we call it ``understood.''
\end{definition}

\section{Summary}

\begin{center}
\fbox{\parbox{0.9\textwidth}{
\textbf{The Master Equation}:
\begin{equation}
\mathrm{Ric}(g) = \nabla\nabla\log p
\end{equation}

\textbf{Unifies}:
\begin{itemize}
    \item Einstein (gravity)
    \item Fisher (information)
    \item Perelman (curvature flow)
    \item Shannon (entropy)
    \item Fokker-Planck (distribution dynamics)
    \item SU(3) (stable triadic distinctions)
\end{itemize}

\textbf{Solutions}:
\begin{itemize}
    \item 1D: exponential
    \item 2D: $g = 1/p$
    \item SU(2): spherical Gaussian
    \item SU(3): QCD vacuum
\end{itemize}
}}
\end{center}
