%==============================================================================
% REFLECTION AND FREEDOM
% The Mathematical Criterion of Consciousness
% Part V, Chapter 3
%==============================================================================

\chapter{Reflection and Freedom}\label{ch:reflection}

\epigraph{Without reflection, the reaction is determined. With reflection, the reaction becomes ambiguous — otherwise it would be an automaton.}{---}

\vnew{This chapter is new in DD v2.0.}

\section{The Core Insight}

\begin{theorem}[Determinism vs. Freedom]
\begin{itemize}
    \item Without reflection: reaction is deterministic
    \item With reflection: reaction becomes non-unique
\end{itemize}
Otherwise it would be an automaton.
\end{theorem}

This is the nucleus of the entire theory of subjectivity — with ultimate precision, without philosophical husk.

\section{Signal in DD}

A signal is simply $\Delta_0(t)$:
\begin{itemize}
    \item Sensory distinction that arrived from outside
    \item It does not ``mean'' anything
    \item It is simply a difference from the previous state
\end{itemize}

Formally:
\[
S(t) = \Delta_0(t)
\]

\section{Automatic Reaction: $F$ Fixed}

When we say: ``receive signal, evaluate against our own, flow with this reaction'' — this is mathematically:
\[
\Delta_1(t) = F(\Delta_0(t))
\]

And $F$ is \textbf{fixed}, meaning:
\begin{itemize}
    \item The same function $F$
    \item With the same $S$
    \item Always gives the same $\Delta_1$ and reaction
\end{itemize}

This is \textbf{determinism}.

No choice. No freedom. No variability.

This is pure biology $\Delta \to F$.

So live:
\begin{itemize}
    \item Reptiles
    \item Basic mammals
    \item Human reflexes
\end{itemize}

This is an automaton.

\section{Where ``Ambiguity'' Appears}

Not ``in feeling,'' not ``in consciousness,'' but \textbf{in the operators that update $F$}.

That is:
\begin{center}
\fbox{Ambiguity = $M$ works}
\end{center}

Formally:
\[
\frac{dF}{dt} = M(\Delta)
\]

In human terms:
\begin{itemize}
    \item The signal reached not only ``evaluation''
    \item But also meta-evaluation of the evaluating function
\end{itemize}

Then:
\[
\Delta_1(t) = F(\Delta_0(t))
\]
but
\[
F(t+1) \neq F(t)
\]

That is, next time you distinguish differently.

This is \textbf{reflection in objective form}.

\section{Non-Philosophical Criterion of Reflection}

\begin{definition}[Reflection]
Reflection begins when the reaction to a signal changes the rules of future reactions.
\end{definition}

Without mysticism.

\textbf{Without reflection:}
\[
\text{signal} \to \Delta \to \text{reaction}
\]

\textbf{With reflection:}
\[
\text{signal} \to \Delta \to \text{reaction} \to \text{change in } F
\]

The reaction doesn't just happen — it \textbf{retrains the system}.

\section{Why Reaction Becomes ``Ambiguous''}

\textbf{Determinism} = $f: \Delta_0 \to \Delta_1$

One function, one result.

\textbf{Ambiguity} = multiple $F$
\[
\Delta_1 = F_i(\Delta_0)
\]

And $F_i$ is chosen dynamically depending on $M$, $H$, $V$ (memory, goal, meta-distinction).

The reaction is not random. It is \textbf{multi-model}.

This is \textbf{variational freedom}.

Technically:
\[
P(\Delta_1 | \Delta_0) > 1
\]

There is a distribution of variants, not a single trajectory.

\section{What This Means in Human Words}

``Ambiguity'' = freedom of maneuver.

This is not chaos, not random, not hesitation:
\begin{itemize}
    \item It is multiplicity of possible $F$
    \item And a meta-decision about which $F$ to apply
\end{itemize}

You described \textbf{metaplasticity} — the main property of consciousness.

\section{Psychology of Response to Stimulus}

\textbf{Without reflection:}
\begin{itemize}
    \item Heard insult $\to$ got offended
    \item Saw danger $\to$ got scared
    \item Felt envy $\to$ aggression
\end{itemize}

\textbf{With reflection:}
\begin{itemize}
    \item Heard insult $\to$ analysis of interpretation methods
    \item Saw danger $\to$ evaluated context
    \item Felt envy $\to$ distinguished values
\end{itemize}

Not the reaction changes, but \textbf{the reflecting space $F$ changes}.

\section{What Really Happens ``When Signal Reaches Reflection''}

\begin{enumerate}
    \item Signal $\Delta_0$ arrives
    \item Multiple different $F$ are applied, not one
    \item Meta-process $M$ chooses the current $F$
    \item This choice updates $F$ in memory ($H$)
\end{enumerate}

That is, each new situation changes the future structure of reactions.

This is directly:
\begin{itemize}
    \item Neuroplasticity (LTP, LTD)
    \item Cognitive restructuring (CBT)
    \item Metacognition (DMN)
\end{itemize}

But DD gives the formula.

\section{The Strongest Thought}

\begin{center}
\fbox{\parbox{0.9\textwidth}{
\textbf{Freedom} = not the ability to choose a reaction, but the ability to change ways of distinguishing \textit{before} the choice.
}}
\end{center}

This is what is usually called ``awareness'' or ``reflection,'' but formally:
\[
\text{Freedom} = \frac{dF}{dt} \neq 0
\]

If $dF/dt = 0$ $\to$ automatic life.

If $dF/dt \neq 0$ $\to$ conscious life.

\section{Final Compressed Explanation}

\begin{theorem}[Minimal Criterion of Consciousness]
Reflection is the appearance of meta-distinction $M$, which makes the reaction not the only possible one.
\end{theorem}

This is the most minimal, rigorous, universal criterion of consciousness.

\section{Clinical Implications}

\begin{definition}[Pathologies in DD terms]
\begin{align}
\text{Depression} &= \Delta \text{ contracted (distinctions shrunk)} \\
\text{Anxiety} &= M \text{ too fast (unstable meta-distinctions)} \\
\text{Trauma} &= \beta\text{-deformation of } M \cdot F \\
\text{Stability} &= \alpha, \beta \text{ in normal range}
\end{align}
\end{definition}

These are states of a dynamical system, not ``diseases of the soul.''
